% Praeamble
\documentclass[a4paper]{IEEEtran}

\usepackage[T1]{fontenc}
\usepackage[ansinew]{inputenc}
\usepackage[ngerman]{babel}

\usepackage{lipsum} % Zum Generieren von Blindtext
\usepackage{csquotes} % Anf�hrungszeichen

% Weitere Symbole
\usepackage{amsmath}
\usepackage{amssymb}

\usepackage{hyperref} % Klickbare Verlinkungen
\usepackage[figure]{hypcap}

\usepackage{listings}
\usepackage{xcolor}

 % Tabelle
\usepackage{tabularx}
\setlength\extrarowheight{4pt}
\usepackage{caption}

\definecolor{codegray}{rgb}{0.5, 0.5, 0.5}
\definecolor{darkgray}{rgb}{0.3, 0.3, 0.3}
\definecolor{backcolor}{RGB}{247, 247, 247}

\lstset{
    language=Python,
    basicstyle=\linespread{1.2}\ttfamily\footnotesize,
    keywordstyle=\color{blue}\ttfamily,
    stringstyle=\color{codegray}\ttfamily,
    commentstyle=\color{teal}\ttfamily,
    morecomment=[l][\color{teal}]{\#},
    backgroundcolor=\color{backcolor},
    rulecolor=\color{codegray},
    showspaces=false,
    showstringspaces=false,
    showtabs=false,
    frame=single,
    numbers=left,
    numberstyle=\scriptsize\color{codegray},
    breaklines=true,
    breakatwhitespace=true,
    captionpos=b
}

% Eingebetteter Code
\usepackage{xparse}
\NewDocumentCommand{\codeword}{v}{%
\texttt{\ttfamily\small\textcolor{darkgray}{#1}}%
}

% Diagramme
\usepackage{tikz}
\usetikzlibrary{shapes, arrows}
\tikzstyle{process} = [rectangle, draw, text centered, minimum height=1.5em]
\tikzstyle{decision} = [diamond, draw, text centered]
\tikzstyle{data}=[trapezium, draw, text centered, trapezium left angle=60, trapezium right angle=120, minimum height=1.5em]
\tikzstyle{connector} = [draw, -latex']

\renewcommand\IEEEkeywordsname{Schl�sselw�rter}
\def\code#1{\texttt{#1}}

% Literaturverzeichnis
\usepackage[style=ieee,backend=biber]{biblatex}
\addbibresource{assets/literature.bib}
\renewcommand*{\bibfont}{\raggedright\small}

\makeatletter
\newcommand*\titleheader[1]{\gdef\@titleheader{#1}}
\AtBeginDocument{%
  \let\st@red@title\@title
  \def\@title{%
    \bgroup\normalfont\large\centering\@titleheader\par\egroup
    \vskip1.5em\st@red@title}
}
\makeatother

% Angaben
\title{Analyse und Implementierung einer Multi-Faktor-Authentifizierung mit \\ Shamir's Secret Sharing}
\titleheader{Moderne Anwendungen der Kryptographie im Sommersemester 2023}
\author{
	\IEEEauthorblockN{Nicolas Proske} \\
	\IEEEauthorblockA{
		\textit{Ostbayerische Technische Hochschule Amberg-Weiden} \\
		E-Mail: \code{n.proske@oth-aw.de} \\
		Matr.-Nr.: \code{87672270}
	}
}

\begin{document}

\maketitle

% Abstract
\begin{abstract}
Heutige Authentifizierungsmethoden beruhen noch oft auf dem Ein-Faktor-Prinzip, wobei dieser eine Faktor, meist ein Passwort, die dahinter liegenden Daten nicht ausreichend vor unautorisiertem Zugriff sch�tzen kann. Im Laufe der Zeit hat sich deshalb die Zwei-Faktor-Authentifizierung etabliert, welche zus�tzlich ein weiteres Merkmal beim Login-Prozess voraussetzt. Diese Studienarbeit betrachtet dar�ber hinaus ein m�gliches Verfahren, wie ein vollumf�nglicher Login-Prozess mit mehreren Faktoren, verteilt auf unterschiedlichen Systemen, mit Hilfe von Multi-Faktor-Authentifizierung und Shamir's Secret Sharing realisiert werden kann.\par

Die Kombination aus einem Passwort, einem biometrischen Merkmal (zum Beispiel einem Fingerabdruck) und einem Wiederherstellungsschl�ssel erh�ht nicht nur die Sicherheit eines Systems, sondern dank dem Einsatz von Secret Sharing auch die Flexibilit�t des Benutzers, da nur zwei der drei Faktoren zur Authentifizierung ben�tigt werden.\par

Die Integration dieser Idee in eine Webanwendung in Verbindung mit einer dazugeh�rigen App zeigt eine sichere M�glichkeit auf, wie ein Login-Prozess in einer modernen Anwendung ablaufen kann.
\end{abstract}

% Schluesselwoerter
\begin{IEEEkeywords}
Authentifizierung, Secret Sharing, Web, App
\end{IEEEkeywords}

\section{Einleitung}
In der heutigen digitalen Welt, in der eine �berw�ltigende Menge an Daten generiert, gespeichert und �ber verschiedene Plattformen �bertragen wird, ist der Bedarf an Datenschutz und Datensicherheit relevanter denn je. Die mit der Digitalisierung einhergehenden M�glichkeiten bergen ein erhebliches Risiko f�r Datendiebstahl, unbefugten Zugriff und Cyberangriffe. Obwohl laut einer Umfrage \autocite[S. 23]{gen-insights-2023} die H�lfte der Erwachsenen weltweit glauben, dass die von ihnen ergriffenen Ma�nahmen ausreichen, um sich gegen Identit�tsdiebstahl zu sch�tzen, sind 63 Prozent dar�ber besorgt, dass ihre Identit�t gestohlen wird. Weiter f�hlen sich knapp sieben von zehn Menschen heute anf�lliger f�r Identit�tsdiebstahl als noch vor ein paar Jahren. Ein wesentlicher Grund f�r die Zunahme von Identit�tsdiebstahl liegt neben zu schwachen Passw�rtern haupts�chlich daran, wie Menschen damit umgehen. Bei einer Frage bez�glich der mehrmaligen Verwendung derselben Benutzernamen und Passw�rter haben 82 Prozent zugegeben, zumindest manchmal dieselben Anmeldedaten f�r unterschiedliche Konten zu verwenden. Knapp die H�lfte davon, etwa 45 Prozent, verwenden sogar immer oder in den meisten F�llen dieselben Zugangsdaten \autocite[S. 12]{ibm-security-2021}.\par

Im Rahmen der vorliegenden Studienarbeit wird eine Analyse und Implementierung einer Multi-Faktor-Authentifizierung mit Shamir's Secret Sharing durchgef�hrt. Dazu wird zu Beginn auf unterschiedliche Authentifizierungsarten eingegangen, einschlie�lich relevanter Fakten sowie der jeweiligen Vor- und Nachteile. In einem weiteren Schritt erfolgt eine kurze Beschreibung der Methode inklusive einer mathematischen Veranschaulichung, um ein gemeinsames Verst�ndnis f�r die nachfolgende Analyse der Implementierung zu erlangen. Daran ankn�pfend f�hrt eine Betrachtung relevanter Sicherheitsaspekte zu einer Zusammenfassung.
\section{Authentifizierung mit Faktoren}
\subsection{Ein-Faktor-Authentifizierung}
Lange Zeit haben Authentifizierungsmethoden auf einem einzelnen Identifikationsfaktor beruht, in der Regel einer Kombination aus Benutzername und Passwort. Wenn diese beiden Parameter �ber mehrere Dienste hinweg identisch sind, bedeutet dies, dass ein Angreifer, der ein einziges Konto kompromittiert, automatisch Zugriff auf die anderen Konten erh�lt \textemdash{} Dabei spielt die St�rke des Passworts keine Rolle. Dieser One-Factor-Authentication (OFA) Ansatz  hat jahrzehntelang das R�ckgrat der Informationssicherheit gebildet. Dennoch hat sich angesichts der zunehmenden Komplexit�t von Cyberbedrohungen gezeigt, dass die Abh�ngigkeit von einem einzigen Faktor f�r die Authentifizierung eine Schwachstelle darstellt, die anf�llig f�r verschiedene Verletzungen wie Brute-Force-Angriffe, Phishing und Keylogging ist. Diese Schwachstellen verdeutlichen, dass OFA f�r heutige Anwendungsf�lle in aller Regel keine ausreichende Sicherheit mehr bietet.

\subsection{Multi-Faktor-Authentifizierung}
Multi-Factor-Authentication (MFA) stellt einen signifikanten Fortschritt in der Evolution der digitalen Sicherheitsma�nahmen dar. Im Gegensatz zur Einzelfaktor-Authentifizierung, die �blicherweise auf einer einzigen Form des Nachweises wie einem Passwort basiert, erh�ht MFA die Sicherheit durch zus�tzliche Schutzebenen, indem mehrere unabh�ngige Zugangsdaten f�r die Authentifizierung erforderlich sind. Diese Zugangsdaten k�nnen in drei Hauptkategorien eingeteilt werden:

\begin{enumerate}
	\item \textit{Wissen}: Informationen, die der Benutzer kennt, wie Passw�rter, PINs und Antworten auf geheime Fragen.
  	\item \textit{Eigenheit}: Biologische Merkmale, die einzigartig f�r den Benutzer sind, wie Fingerabdr�cke, Netzhautmuster oder Gesichtserkennung.
  	\item \textit{Besitz}: Gegenst�nde oder Ger�te, die der Benutzer besitzt, wie Smartphones, Chipkarten oder physische Schl�ssel. Die Best�tigung des Besitzes kann verschiedene Formen annehmen, angefangen von der Entgegennahme und Eingabe eines per SMS an eine registrierte Telefonnummer gesendeten Codes bis hin zum Einsetzen eines physischen Schl�ssels in ein Schloss.
\end{enumerate}

Der Hauptvorteil von MFA gegen�ber OFA liegt daher im schichtbasierten Ansatz. Selbst wenn ein Angreifer es schafft, einen Authentifizierungsfaktor zu umgehen, bieten die verbleibenden Faktoren weiterhin Schutz. Eine Kompromittierung eines Faktors gef�hrdet also nicht die Gesamtsicherheit. Trotz der St�rken bringt eine MFA auch eigene Herausforderungen mit sich, wie beispielsweise die potenziell erh�hte Komplexit�t und die Notwendigkeit f�r Benutzer, mehrere Authentifizierungsfaktoren zu verwalten. Dennoch �berwiegen die Vorteile der MFA oft diese potenziellen Nachteile, insbesondere in Umgebungen, in denen der Schutz sensibler Daten oberste Priorit�t hat.
\section{Shamir's Secret Sharing}
Secret Sharing ist ein grundlegender Baustein der modernen Kryptographie. Eines der bekanntesten Verfahren wurde am 1. November 1979 ver�ffentlicht und ist nach seinem Erfinder Adi Shamir, einem israelischen Kryptographen, benannt: Shamir's Secret Sharing \autocite{shamir-secretsharing-1979}.\par

Es basiert auf der Idee, ein Geheimnis in mehrere Teile, sogenannte Shares, aufzuteilen. Um das Geheimnis wiederherzustellen, m�ssen eine bestimmte Anzahl dieser Shares zusammengebracht werden. Jeder einzelne Share ist f�r sich genommen bedeutungslos und gibt keinerlei Informationen preis. Ein Schwellenwert definiert die minimale Anzahl von Shares, die erforderlich sind, um das Geheimnis wiederherstellen zu k�nnen. Dies stellt sicher, dass das Geheimnis selbst dann sicher bleibt, wenn ein Teil der Shares verloren gehen oder in die H�nde eines Angreifers gelangen. Das dabei verwendete (k, n)-Schwellenwertschema legt fest, wie viele $k$ Shares ben�tigt werden, um auf das Geheimnis zu kommen, $n$ ist gr��er $k$ und bezieht sich auf die Gesamtzahl der Shares, in die das Geheimnis aufgeteilt wird.

\subsection{Mathematische Veranschaulichung}
Shamir's Secret Sharing basiert auf dem Prinzip der Polynominterpolation in endlichen K�rpern, wobei $k$ Punkte ein Polynom vom Grad $k - 1$ eindeutig definieren. Um dies anhand eines mathematischen Beispiels zu veranschaulichen, wird im Folgenden ein (2, 3)-Schwellenschema mit $k = 2$ und $n = 3$ betrachtet, bei dem das Geheimnis $S$ der Zahl $42$ entspricht. Sei $p = 43$ eine Primzahl mit $p > S$. Alle Berechnungen erfolgen im endlichen K�rper $\mathbb{F}_p$.

\subsubsection{Generierung der Shares}
Der erste Schritt besteht darin, ein Polynom vom Grad $k - 1 = 2 - 1 = 1$ aufzustellen:

\begin{equation*}
f(x) = mx + b \mod{p}
\end{equation*}

Die Konstante $b$ entspricht dabei dem Geheimnis $S$. Aus Gr�nden der �bersichtlichkeit wird in diesem Beispiel $m = 4$ gew�hlt:

\begin{equation*}
f(x) = 4x + 42 \mod{43}
\end{equation*}

Im n�chsten Schritt erfolgt die Berechnung von $n$ Punkten in der Ebene. Dazu wird f�r $x = 1...n$ eingesetzt:

\begin{equation*}
\begin{aligned}
&\text{F�r } x = 1: y_1 = f(1) = 4*1 + 42 \mod{43} = 3 \\
&\text{F�r } x = 2: y_2 = f(2) = 4*2 + 42 \mod{43} = 7 \\
&\text{F�r } x = 3: y_3 = f(3) = 4*3 + 42 \mod{43} = 11
\end{aligned}
\end{equation*}

Jeder der berechneten Punkte $(1, 3), (2, 7), (3, 11)$ repr�sentiert dabei einen Share. \autoref{tab:shares} zeigt die Verteilung aller drei Shares an insgesamt drei unterschiedliche Nutzer:

\begin{table}[!h]
\centering
\normalsize
\begin{tabularx}{0.45\textwidth}{*3{>{\centering\arraybackslash}X}@{}}
\textit{Verteilung an} user($x$) & \textit{$f(x)$} & share($x, f(x) \mod{p}$) \\
\hline
1 & $f(1) = 46$ & (1, 3) \\
2 & $f(2) = 50$ & (2, 7) \\
3 & $f(3) = 54$ & (3, 11) \\
\end{tabularx}
\medskip
\caption{Verteilung der Shares}
\label{tab:shares}
\end{table}

\subsubsection{Rekonstruktion mit linearem Gleichungssystem}
Sind nun $k$ Punkte gegeben, kann das urspr�ngliche Geheimnis rekonstruiert werden. Im Folgenden werden die Punkte $(1, 3)$ und $(2, 7)$ als Gleichungen in einem linearen Gleichungssystem dargestellt:

\begin{equation*}
\begin{aligned}
&\text{1. } m + b = 3 \\
&\text{2. } 2m + b = 7
\end{aligned}
\end{equation*}

Die Unbekannten werden nun �ber das Substitutionsverfahren gel�st. Durch Umstellen der ersten Gleichung nach $b$ folgt $b = 3 - m$. Dieser Ausdruck wird in die zweite Gleichung eingesetzt, was zu $2m + (3 - m) = 7$ f�hrt. Daraus folgt $m = 4$. Die erhaltene L�sung f�r $m$ wird dann in die umgestellte erste Gleichung eingesetzt, um $b$ zu berechnen: $b = 3 - 4 = -1$. Da die Berechnungen im endlichen K�rper $\mathbb{F}_{43}$ durchgef�hrt werden, wird das Ergebnis $\mod{43}$ genommen, um das Geheimnis im Wertebereich von $0$ bis $p-1$ zu erhalten: $b = -1 \mod{43} = 42$, was dem Geheimnis $S = 42$ entspricht. Bei gr��eren Werten von $k$ w�rde ein Polynom h�heren Grades und ein entsprechend gr��eres lineares Gleichungssystem entstehen.

\subsubsection{Rekonstruktion mit Lagrange-Interpolations-Formel}
Die Lagrange-Interpolation ist das in der Praxis am h�ufigsten eingesetzte Verfahren zur Bestimmung des Polynoms einer bestimmten Ordnung, das durch eine gegebene Menge von Punkten verl�uft. Diese Methode bietet den Vorteil, dass sie direkt eine Formel zur Rekonstruktion des Geheimnisses liefert, ohne dass ein Gleichungssystem explizit gel�st werden muss. Die Koeffizienten $m_0, ..., m_{k-1}$ eines unbekannten Polynoms $f$ vom Grad $k-1$ aus $k$ Punkten $(x_i, y_i)$ k�nnen wie folgt berechnet werden:

\[
f(x) = \sum_{i=1}^{k} \left[ y_i \cdot \prod_{\substack{1 \leq j \leq k \\ i \neq j}} \frac{x - x_j}{x_i - x_j} \right] \mod{p}
\]

Unter Verwendung dieser Formel l�sst sich $m_0 = f(0)$ und damit das Geheimnis $S$ aus $k$ gegebenen Punkten berechnen \autocite[S. 65 f.]{bsi-richtlinie-2023}.
\section{Die Idee der Kombination}
Das Ziel dieser Arbeit besteht darin, die Vorteile beider Verfahren zu kombinieren, um eine robuste und sichere Authentifizierungsmethode zu entwickeln.

\begin{figure}[!h]
\centering
\begin{tikzpicture}

\node [data, fill=gray!10] at (0,-1) (input_factors) {Nutzer legt drei Faktoren fest};
\node [process, fill=gray!10] at (0,-2) (transform_factors) {Faktoren in Geheimnis umwandeln};
\node [process, fill=gray!10] at (0,-3) (gen_shares) {Geheimnis in Shares aufteilen};
\node [process, fill=gray!10] at (0,-4) (store_shares) {Shares verteilt speichern};

\path [connector] (input_factors) -- (transform_factors);
\path [connector] (transform_factors) -- (gen_shares);
\path [connector] (gen_shares) -- (store_shares);

\end{tikzpicture}
\caption{�berblick: Generierung der Shares}
\label{fig:ueberblick_gen_shares}
\end{figure}

\autoref{fig:ueberblick_gen_shares} zeigt den groben Ablauf, wie die Shares erzeugt werden. Im Kern wird ein Geheimnis aus mehreren Authentifizierungsfaktoren generiert und mithilfe von Shamir's Secret Sharing aufgeteilt. Die zur Multi-Faktor-Authentifizierung verwendeten Faktoren entsprechen einem Passwort (Wissen), dem Fingerabdruck des Nutzers (Eigenheit) und einem Wiederherstellungsschl�ssel in Form eines QR-Codes (Besitz). Die Kombination aller drei Faktoren dient als Grundlage zur Berechnung des Geheimnisses, welches anschlie�end mittels Shamir's Secret Sharing in drei Shares aufgeteilt wird, wobei nur zwei St�ck zur sp�teren Authentifizierung ben�tigt werden. Jeder Share wird anschlie�end eindeutig zu einem der obigen Faktoren zugeordnet und auf unterschiedlichen Wegen sicher abgelegt, zum Beispiel der Passwort-Share in einer Datenbank, der Fingerabdruck-Share im sicheren Bereich eines Smartphones und der zugeh�rige Share zum Wiederherstellungsschl�ssel als QR-Code ausgedruckt an einem sicheren Ort.\par

Nachdem alle Shares erfolgreich generiert wurden und gespeichert sind, kann sich der Nutzer mit zwei der drei zu Beginn festgelegten Faktoren authentifizieren. Mit Blick auf \autoref{fig:ueberblick_auth_shares} w�hlt der Nutzer zu Beginnn aus, welche zwei Faktoren er dazu verwenden m�chte. Nach Eingabe eines korrekten Faktores gibt das System den verkn�pften Share frei, welcher im Anschluss zur Rekonstruktion verwendet wird. Sobald beide Shares zur Verf�gung stehen wird das Geheimnis wiederhergestellt und �berpr�ft, ob der Wert dem urspr�nglichen Geheimnis entspricht. Falls ja, ist die Authentifizierung erfolgreich, andernfalls erh�lt der Nutzer eine Fehlermeldung.

\begin{figure}[!h]
\centering
\begin{tikzpicture}

\node [data, fill=gray!10] at (0,-1) (def_factors) {Nutzer w�hlt zwei Faktoren aus};
\node [data, fill=gray!10] at (0,-2) (input_factors) {Nutzer gibt Faktoren ein};
\node [process, fill=gray!10] at (0,-3) (recon_shares) {Geheimnis rekonstruieren};
\node [decision, fill=gray!10] at (0,-5) (correct) {Korrekt?};

\node[draw=none] at (1.625, -4.75) (no) {Nein};
\node[draw=none] at (0.35, -6.25) (yes) {Ja};

\node [process, fill=red!30] at (3,-5) (error) {Fehler};
\node [process, fill=green!30] at (0,-7) (success) {Authentifiziert};

\path [connector] (def_factors) -- (input_factors);
\path [connector] (input_factors) -- (recon_shares);
\path [connector] (recon_shares) -- (correct);
\path [connector] (correct) -- (error);
\path [connector] (correct) -- (success);

\end{tikzpicture}
\caption{�berblick: Authentifizierung mit Shares}
\label{fig:ueberblick_auth_shares}
\end{figure}

\subsection{Vorteile}
\subsubsection{Erh�hte Sicherheit}
MFA und SSS erg�nzen sich gegenseitig, um eine robuste Sicherheitsarchitektur zu schaffen. W�hrend MFA bereits eine zus�tzliche Sicherheitsebene durch die Verwendung mehrerer Faktoren bietet, stellt Shamir's Secret Sharing sicher, dass die zu sch�tzenden Daten unverschl�sselt dezentral gespeichert werden k�nnen, da ein einzelner Share keine R�ckschl�sse auf das Geheimnis zul�sst. Dadurch wird das Risiko eines vollst�ndigen Datenlecks oder unbefugten Zugriffs drastisch minimiert.

\subsubsection{Flexibilit�t}
Benutzer k�nnen aus drei verschiedenen Optionen (Passwort, Fingerabdruck und Wiederherstellungsschl�ssel) zur Authentifizierung w�hlen. Dar�ber hinaus kann die Aufteilung von Shares an unterschiedliche Ger�te oder Personen erfolgen, um sowohl den Bed�rfnissen und Anforderungen des Nutzers als auch eines Systems gerecht zu werden.

\subsubsection{Schutz vor Datenverlust}
Wenn beispielsweise ein Benutzer das verkn�pfte Smartphone verliert oder es ir�re�pa�ra�bel besch�digt wurde, ist weiterhin ein Zugriff �ber die beiden anderen Faktoren gew�hrleistet. Dies bietet einen zus�tzlichen Schutz vor Datenverlust.\par

Zusammenfassend l�sst sich sagen, dass das Zusammenspiel beider Methoden die Sicherheit eines Authentifizierungsprozesses erh�ht. Deshalb wird im Folgenden eine m�gliche Implementierung vorgestellt, welche die einzelnen Schritte im Detail verdeutlichen soll.
\section{Implementierung}
In diesem Kapitel wird die Idee der Kombination von MFA und SSS anhand einer Implementierung in Python v3.10.9 schrittweise erkl�rt. Die folgenden Code-Ausschnitte dienen nur zur Veranschaulichung und sind deshalb ohne weitere Vorkehrungen nicht zwingend lauff�hig.

\subsection{Phase 1: Konstruktion des Geheimnisses}
Bevor eine Authentifizierung stattfinden kann, m�ssen die daf�r ben�tigten Shares erst einmal erzeugt werden. Dazu wird ein Geheimnis ben�tigt, welches auf allen drei Faktoren basiert. Das Ziel dieser Phase ist es, eine nat�rliche Zahl zu konstruieren, die als Geheimnis f�r Shamir's Secret Sharing verwendet werden kann.

\subsubsection{Nutzer legt drei Faktoren fest}
Zu den eben genannten Faktoren z�hlen ein Passwort, ein Fingerabdruck und ein Wiederherstellungsschl�ssel. Wie in \autoref{code:input_factors} abgebildet m�ssen die ersten beiden Faktoren vom Nutzer eingegeben werden, der Dritte wird zuf�llig in Form eines 16 Byte langen hexadezimalen Strings generiert.

\begin{lstlisting}[caption={Initialisierung der drei Faktoren},label=code:input_factors,numbers=right]
password = input() # 1. Faktor
fingerprint = input() # 2. Faktor
recovery_key = os.urandom(16).hex() # 3. Faktor
\end{lstlisting}

\subsubsection{Faktoren umwandeln}
All diese Faktoren werden im sp�teren Verlauf f�r die Authentifizierung ben�tigt. Daher ist es wichtig, dass diese Informationen umgewandelt werden, um m�gliche R�ckschl�sse auf die urspr�nglichen Eingaben des Nutzers auszuschlie�en. Aus diesem Grund werden im n�chsten Schritt alle Faktoren mit SHA256 gehasht (siehe \autoref{fig:hash_factors}).

\begin{figure}[!h]
\centering
\begin{tikzpicture}

\node [data, fill=orange!10] at (-2.5,0) (password) {Passwort};
\node [data, fill=blue!10] at (-0,0) (fingerprint) {Fingerabdruck};
\node [data, fill=green!10] at (2.5,0) (recovery_key) {Schl�ssel};
\node [data, fill=gray!10] at (0,-1.25) (hash_factors) {SHA256-Hash};
\node [data, fill=orange!10] at (-2.5,-2.25) (password_hash) {807A09...};
\node [data, fill=blue!10] at (-0,-3) (fingerprint_hash) {8587EC...};
\node [data, fill=green!10] at (2.5,-2.25) (recovery_key_hash) {39B997...};

\node[draw=none] at (-3.25, -0.75) {W@6a...};
\node[draw=none] at (0.6, -0.625) {1011...};
\node[draw=none] at (3.1, -0.75) {cf61...};

\path [connector] (password) |- (hash_factors);
\path [connector] (fingerprint) -- (hash_factors);
\path [connector] (recovery_key) |- (hash_factors);
\path [connector] (hash_factors) |- (password_hash);
\path [connector] (hash_factors) -- (fingerprint_hash);
\path [connector] (hash_factors) |- (recovery_key_hash);

\end{tikzpicture}
\caption{Faktoren umwandeln}
\label{fig:hash_factors}
\end{figure}

Die in \autoref{code:hash_factors} gegebene Funktion \texttt{hash\_string} nimmt einen String \texttt{value} entgegen. Die Funktion greift auf die \texttt{hashlib}-Bibliothek zur�ck, um den SHA-256-Hash des gegebenen Werts zu berechnen. Zun�chst wird der �bergebene Wert mit \texttt{.encode()} in eine Bytefolge umgewandelt, woraufhin  die \texttt{.digest()}-Methode auf den berechneten Hash angewendet wird, um das Ergebnis als Bytes zur�ckzugeben, um diese im nachfolgenden Schritt in eine ganze Zahl umwandeln zu k�nnen.

\begin{lstlisting}[caption={Hashen der drei Faktoren},label=code:hash_factors,numbers=left]
def hash_string(value):
    return hashlib.sha256(value.encode()).digest()

password_hash = hash_string(password)
fingerprint_hash = hash_string(fingerprint)
recovery_key_hash = hash_string(recovery_key)
\end{lstlisting}

\subsubsection{Hashwerte in Zahlen umwandeln}
Alle drei erhaltenen Hashwerte m�ssen nun als Zahlen interpretiert werden, da Shamir's Secret Sharing eine ganze Zahl f�r das Geheimnis fordert. Die Funktion \texttt{hash\_to\_int} aus \autoref{code:interpret_hashes} nimmt ebenfalls einen Wert \texttt{value} entgegen, der hier allerdings die zuvor generierte Bytefolge repr�sentiert. Durch Verwendung der Methode \texttt{int.from\_bytes()} mit dem Parameter \texttt{value} wandelt die Funktion diese Bytefolge in eine Ganzzahl um. Dabei erfolgt die Interpretation der Bytes in der Reihenfolge \textquote{big}, wodurch das Most Significant Bit zuerst und das Least Significant Bit zuletzt ber�cksichtigt wird. Das Ergebnis, also die umgewandelte Ganzzahl, wird von der Funktion zur�ckgegeben.

\begin{lstlisting}[caption={Hashwerte als Zahlen interpretieren},label=code:interpret_hashes,numbers=left]
def hash_to_int(value):
    return int.from_bytes(value, byteorder="big")

password_number = hash_to_int(password_hash)
fingerprint_number = hash_to_int(fingerprint_hash)
recovery_key_number = hash_to_int(recovery_key_hash)
\end{lstlisting}

\subsubsection{Geheimnis erzeugen}
Der vorletzte Schritt besteht darin, alle Zahlen zusammenzuf�gen, um das Geheimnis zu erhalten. Dabei ist zu beachten, dass die Zahlen nicht addiert, sondern in zuf�lliger Reihenfolge konkateniert werden. Dazu wird eine Liste mit den Zahlen der drei umgewandelten Hashwerte erstellt und anschlie�end gemischt. Die nun zuf�llig angeordneten Zahlen werden in einer Schleife durchlaufen, in einen String umgewandelt und aneinandergereiht. Dieser String wird zum Schluss wieder zu einem Integer konvertiert und der Variable f�r das Geheimnis zugewiesen (siehe \autoref{code:concat_numbers}). 

\begin{lstlisting}[caption={Zahlen zu Geheimnis konkatenieren},label=code:concat_numbers,numbers=right]
numbers = [password_number, fingerprint_number, recovery_key_number]
random.shuffle(numbers)
S = int("".join(str(num) for num in numbers))
\end{lstlisting}

Um das Geheimnis w�hrend der Authentifizierung bei einer erfolgreichen Rekonstruktion auf Korrektheit pr�fen zu k�nnen, muss es gespeichert werden. Dazu wird es wie die Faktoren in \autoref{code:hash_factors} mit SHA256 gehasht.

\subsection{Phase 2: Generierung der Shares}
Das nicht gehashte Geheimnis wird in dieser Phase dazu ben�tigt, um es unter Verwendung von Shamir's Secret Sharing in einzelne Shares aufteilen zu k�nnen.

\subsubsection{Primzahl erzeugen}
Alle Berechnungen erfolgen wie auch zu Beginn in der mathematischen Veranschaulichung in einem endlichen K�rper. Dieser wird definiert als \( GF(p) \), wobei \( p \) eine Primzahl in derselben Gr��enordnung des Geheimnisses ist. Die Bibliothek \textquote{sympy} bietet daf�r die Funktion \textquote{nextprime}. Die Eingabe der Variable \( S \) erzeugt die n�chstgr��ere Primzahl, welche f�r die nachfolgenden Berechnungen verwendet wird.

\subsubsection{Shares erzeugen}
Nach allen Vorbereitungen wird das Geheimnis nun im letzten Schritt in einzelne Shares aufgeteilt. Das in \autoref{code:create_shares} verwendete (2, 3)-Schwellenwertschema erzeugt insgesamt drei Shares, wovon zwei zur Rekonstruktion ben�tigt werden.

\begin{lstlisting}[caption={Shares erzeugen},label=code:create_shares,numbers=right]
def create_shares(S, p):
    m = int.from_bytes(os.urandom(32), byteorder="big") # Zuf�lliger Koeffizient
    shares = []
    for x in range(1, 4): # x aufsteigend iterieren
        y = (m * x + S) % p # y-Wert berechnen
        shares.append((x, y)) # Punkt hinzuf�gen
    return shares

shares = create_shares(S, p)
\end{lstlisting}
\section{Relevante Sicherheitsaspekte}
Die Implementierung dient dazu, die Idee hinter der Kombination von MFA und SSS anschaulich zu vermitteln. Aus diesem Grund werden dort bestimmte Sicherheitsaspekte au�en vor gelassen, um den Quelltext �bersichtlich und leicht verst�ndlich zu halten. Im Folgenden wird daher auf relevante Kriterien eingegangen, die in einer Realisierung beachtet werden sollten.

\subsection{Randomisierung zur Geheimniserzeugung}
Das hier implementierte Geheimnis setzt sich einem Passwort, einem Fingerabdruck und einem zuf�llig generierten Wiederherstellungeschl�ssel zusammen. Die darauf berechneten Hashwerte werden als Zahlen interpretiert und zuf�llig aneinandergereiht, wodurch sich das Geheimnis ergibt. Dieser Ansatz ist nur sicher, solange ein Angreifer keinen Zugriff auf alle Hashwerte (oder die Faktoren selbst oder eine Mischung aus beidem) hat. Da dies nur in der Theorie immer der Fall ist, m�ssen zus�tzliche Sicherheitsebenen geschaffen werden, um das Geheimnis zu sch�tzen. Ist ein Angreifer in Besitz aller Hashwerte, kann das Geheimnis so in wenigen Schritten rekonstruiert werden, da nach dem Prinzip von Kerckhoffs die Sicherheit eines Verfahren von der Geheimhaltung der Schl�ssel, hier der Shares, abh�ngt und nicht von der Geheimhaltung des Algorithmus \textemdash{} Ein Angreifer kennt daher den Algorithmus und somit das Vorgehen zur Berechnung des Geheimnisses.\par

Die Randomisierung ist im Beispiel dieser Implementierung in Form der zuf�lligen Aneinanderreihung der als Zahlen interpretierten Hashwerte angedeutet. Durch Ausprobieren ben�tigt ein Angreifer bei drei Zahlen im schlechtesten Fall jedoch nur sechs Versuche, um alle m�glichen Kombinationen auszuprobieren und das Geheimnis zu erhalten. F�r die erste Zahl gibt es $n$ M�glichkeiten, f�r die zweite Zahl (da die erste Zahl bereits ausgew�hlt wurde) $n-1$ M�glichkeiten und f�r die dritte Zahl (da bereits zwei Zahlen ausgew�hlt wurden) $n-2$ M�glichkeiten. Die Anzahl der M�glichkeiten f�r die drei gegebenen Faktoren betr�gt dann $n * (n-1) * (n-2) = n! = 3! = 6$. Um dies zu verhindern, werden im Folgenden m�gliche L�sungsans�tze vorgeschlagen:

\begin{enumerate}
	\item Salt beim Hashing verwenden: Durch das Hinzuf�gen einer zuf�llig gew�hlten Zeichenfolge (Salt) an jeden Faktor ist es einem Angreifer nicht mehr m�glich, den ben�tigten Hashwert nur auf Basis des Faktors (d. h. ohne Kenntnisse �ber den Salt) zu berechnen.
	\item Zahlen mit Padding auff�llen: Aufgrund der Verwendung von SHA-256 entspricht der Hashwert und somit auch die daraus abgeleitete Zahl der Gr��enordnung von 256 Bit. Aus sicherheitstechnischen Gr�nden macht es durchaus Sinn, diese Bitl�nge �ber das Hinzuf�gen eines Paddings zu erh�hen. Einem Angreifer ist es dadurch unm�glich, im Nachhinein das Geheimnis zu ermitteln, selbst wenn alle Faktoren bekannt sind. Zudem kann �ber ein Padding die Gesamtl�nge des Geheimnisses gesteuert werden. Je l�nger das Geheimnis ist, desto gr��er muss die gew�hlte Primzahl sein. Auf die Bedingungen und Auswirkungen dieser Primzahl wird im n�chsten Abschnitt eingegangen.
\end{enumerate}

Beim Thema Randomisierung ist zudem wichtig zu erw�hnen, dass in einer realen Anwendung die Koeffizienten $m_i$ f�r $i > 0$ echt zuf�llig und entsprechend der Gleichverteilung aus $\mathbb{F}_p$ gew�hlt werden m�ssen (vgl. Zeile 2 in \autoref{code:create_shares}).

\subsection{Wahl der richtigen Primzahl}
Die Anforderung an die Primzahl $p \geq max(2*r, n + 1)$, wobei $r$ die Bitl�nge des Geheimnisses $S$ repr�sentiert, stellt sicher, dass das Sicherheitsniveau des Verfahrens mindestens die Bitl�nge des zu sch�tzenden Geheimnisses ist und $n$ Shares daraus erzeugt werden k�nnen. Weiter erreicht das Secret-Sharing-Schema von A. Shamir laut dem Bundesamt f�r Sicherheit in der Informationstechnik (BSI) informationstheoretische Sicherheit, was bedeutet, dass ein Angreifer mit unbegrenzten Ressourcen nicht in der Lage ist, das Geheimnis ohne Kenntnis �ber alle $k$ Shares zu rekonstruieren. Die Sicherheit des Verfahrens ist daher nur von der Geheimhaltung der Shares abh�ngig, daher muss "[j]egliche Kommunikation �ber die Teilgeheimnisse [...] verschl�sselt und authentisiert stattfinden, soweit es einem Angreifer physikalisch m�glich ist, diese Kommunikation aufzuzeichnen oder zu manipulieren." \autocite[S. 66]{bsi-richtlinie-2023}.

\subsection{Umgang mit kritischen Daten}
Der Begriff \textquote{Umgang} bezieht sich insbesondere auf die Eingabe, Verarbeitung, �bertragung und Speicherung von sch�tzenswerten Daten. Zu diesen kritischen Daten geh�ren:

\begin{enumerate}
	\item Faktoren (Passwort, Fingerabdruck und Wiederherstellungsschl�ssel): Diese Daten werden f�r die Erstellung des Geheimnisses und zur Authentifizierung ben�tigt. Erh�lt ein Angreifer Zugang zu zwei der drei Informationen, kann dieser sich authentifizieren.
	\item Geheimnis: �hnlich wie bei den Faktoren ist auch das Geheimnis ein kritisches Datum. Durch Verwendung von $k-1$ Anteilen und des Geheimnisses ist es m�glich, das Polynom eindeutig zu rekonstruieren. Zudem k�nnen mithilfe dieser Informationen alle restlichen Anteile generiert werden, sofern die $x$-Koordinaten bekannt sind. Falls jedoch weniger als $k-1$ Anteile vorhanden sind, l�sst sich das Polynom selbst mit dem Geheimnis nicht rekonstruieren.
	\item Shares: Wie im letzten Abschnitt bereits durch den BSI erw�hnt, gilt es die Shares geheim zu halten. Um sicherzugehen, dass eine Nachricht nicht w�hrend der �bertragung von dem erwarteten Sender stammt und diese nicht auf dem Weg manipuliert wurde, k�nnen beispielsweise Message Authentication Codes (MAC) oder Signaturverfahren eingesetzt werden. 
\end{enumerate}

Bei all diesen Daten gilt es zu beachten, dass die damit verbundenen Informationen vor einer �bertragung oder Speicherung unkenntlich gemacht werden. Dies kann zum Beispiel durch die Anwendung einer Hashfunktion oder einer Verschl�sselung passieren.

\subsection{Verwendung einer kryptographisch starken Hashfunktion}
Die in der Implementierung verwendete Hashfunktion SHA-256 ist ein Beispiel einer kryptographisch starken Hashfunktion. Kryptographisch stark bedeutet laut einer Definition des BSI \autocite[S. 46]{bsi-richtlinie-2023}, dass es praktisch nicht m�glich ist ...

\begin{itemize}
	\item ... f�r ein gegebenes $h \in \lbrace 0, 1 \rbrace ^n$ einen Wert $m \in \lbrace 0, 1 \rbrace ^*$ mit $H(m) = h$ zu finden (\textit{Einweg-Eigenschaft}).
	\item ... f�r ein gegebenes $m \in \lbrace 0, 1 \rbrace ^*$ einen Wert $m' \in \lbrace 0, 1 \rbrace ^* \textbackslash \lbrace m \rbrace $ mit $H(m) = H(m')$ zu finden (\textit{2nd-Preimage-Eigenschaft}).
	\item ... zwei Werte $m, m' \in \lbrace 0, 1 \rbrace ^*$ mit $m \neq m'$ und $H(m) = H(m')$ zu finden (\textit{Kollisionsresistenz}).
\end{itemize}

Weitere Hashfunktionen, die diese Eigenschaften nach heutigem Kennt�nis�stand erf�llen, sind SHA-256, SHA-512/256, SHA-384 und SHA-512 sowie die SHA-3-Familie ab SHA3-256 und k�nnen f�r eine Realisierung verwendet werden. Die Verwendung einer anderen Hashfunktion als SHA-256 wirkt sich auf die Laufzeit aus, da eine Verdopplung der Bitl�nge (SHA-256 vs. SHA-512) der einzelnen Hashwerte auch zu einer Verdopplung der Bitl�nge des Geheimnisses f�hrt, wodurch die Bitl�nge der zu berechnenden Primzahl ebenfalls doppelt so gro� sein muss. Die konkreten Unterschiede werden im n�chsten Kapitel genauer betrachtet.
\section{Benchmarks}
\lipsum[1]

% Ausprobieren:
%	- SHA512 statt SHA256
%	- Gro�e Primzahl (3000 Bits)
\section{Anwendung in der digitalen Welt}
Die Anwendung des Domain-Name-Systems (DNS) hat eine zentrale Bedeutung in der digitalen Welt erlangt, da es die Br�cke zwischen menschenlesbaren Domainnamen wie \codeword{oth-aw.de} und maschinenlesbaren IP-Adressen wie \codeword{195.37.42.173} bildet. Ohne ausreichende Sicherheitsvorkehrungen besteht f�r einen Angreifer die M�glichkeit, sich als ein solches DNS-System auszugeben und eine falsche IP-Adresse f�r eine bestimmte Domain zur�ckzuliefern, was erhebliche Folgen haben kann. Um dies zu verhindern, nutzt die daf�r verantwortliche Non-Profit-Organisation ICANN Kryptographie. ICANN greift dabei auf das Prinzip von Secret Sharing zur�ck, indem der Master-Schl�ssel in insgesamt sieben Shares aufgeteilt und auf Smartcards an sieben Personen mit unterschiedlichen geografischen Standorten verteilt ist. Das dabei verwendete (5, 7)-Schwellenwertschema sagt aus, dass f�nf der sieben Personen zusammenkommen m�ssen, um auf das Geheimnis, den Master-Schl�ssel, zugreifen zu k�nnen \autocite{rosulek-icann-2017}. Dieses Anwendungsszenario zeigt das Vertrauen in Secret Sharing und best�tigt, dass sicherheitsrelevante Dienste von diesem Verfahren profitieren k�nnen.

\subsection{Beispiel anhand einer modernen Webanwendung}
Eine Besonderheit der obigen Implementierung ist, dass der Benutzer zu keinem Zeitpunkt direkten Zugriff auf die Shares hat. Jeder Share wird erst nach der korrekten Eingabe des dazugeh�rigen Faktors \textquote{freigegeben}. In diesem Abschnitt wird eine m�gliche Umsetzung dieses Konzepts unter Ber�cksichtigung solcher Besonderheiten beschreiben.\par

Eines der Hauptmerkmale von Shamir's Secret Sharing ist es, dass $n$ Shares auf genauso viele Instanzen, zum Beispiel Personen, verteilt werden. Die Multi-Faktor-Authentifizierung sagt aus, dass ein Login-Prozess mit mehreren unabh�ngigen Zugangsdaten der Kategorien Wissen, Eigenheit und Besitz die Sicherheit dessen erh�ht. Eine Kombination aus beiden Ans�tzen ergibt das in dieser Studienarbeit beschriebene Verfahren: Drei Faktoren, ein Passwort (Wissen), ein Fingerabdruck (Eigenheit) und ein Wiederherstellungsschl�ssel (Besitz), die sowohl f�r die Erzeugung der Shares als auch zur sp�teren Authentifizierung ben�tigt werden. Um dies als alltagstaugliches Verfahren umsetzen zu k�nnen, werden als Instanzen keine Personen, sondern unterschiedliche Systeme in Form einer Webanwendung mit einer dazugeh�rigen App eingesetzt. Der im Folgenden beschriebene Ablauf (Happy Path) bezieht sich daher auf die vollumf�ngliche Authentifizierung in einer Webanwendung.

\subsubsection{Erstellung eines Kontos}
Jeder Benutzer, der die Webanwendung verwenden m�chte, ben�tigt ein Konto. Bei der Registrierung muss neben einer eindeutigen Kennung (z. B. E-Mail-Adresse oder Benutzername) nur ein Passwort (1. Faktor) festgelegt werden. Optional wird durch Plausibilit�tspr�fungen sichergestellt, dass die Eingabe den Mindestanforderungen f�r sichere Passw�rter \autocites{bsi-passwoerter-2023} entspricht:

\begin{itemize}
	\item L�nge mindestens acht, besser zw�lf, Zeichen
	\item Vier verschiede Zeichenarten (Gro�- und Kleinbuchstaben, Zahlen und Sonderzeichen)
	\item Zuf�llige Aneinanderreihung der Zeichen
\end{itemize}

Nach Fertigstellung ist der Benutzer tempor�r auf der Webanwendung angemeldet und wird darauf hingewiesen, sich in der dazugeh�rigen App mit den eben festgelegten Zugangsdaten anzumelden. Zu diesem Zeitpunkt ist der Zugriff auf die Anwendung allerdings noch eingeschr�nkt.\par

Die App muss nun auf dem Smartphone des Nutzers installiert und ge�ffnet sein. Nach dortiger Anmeldung mit E-Mail/Benutzername und Passwort wird der Benutzer aufgefordert, ein biometrisches Merkmal (2. Faktor), zum Beispiel einen Finger- oder Gesichtsabdruck, zu hinterlegen. Ist dieser Schritt erfolgt, wird abschlie�end ein zuf�lliger Wiederherstellungsschl�ssel (3. Faktor) generiert und dem Benutzer angezeigt. Dieser hat die M�glichkeit, diesen entweder direkt abzuschreiben oder alternativ als QR-Code zu speichern (um diesen beispielsweise ausdrucken zu k�nnen). Nach der Best�tigung, dass der Wiederherstellungsschl�ssel sicher abgelegt worden ist, muss dieser zur Verifikation erneut eingegeben werden. Danach ist der Login-Prozess aus Benutzersicht abgeschlossen.

\subsubsection{Zuordnung und Speicherung der Shares}
Im Hintergrund beginnt nach Abschluss der Verifikation des dritten Faktors die Generierung des Geheimnisses inklusive der Aufteilung in Shares, welche abschlie�end den einzelnen Faktoren zugewiesen werden. Aus Gr�nden der �bersichtlichkeit wird in diesem Beispiel davon ausgegangen, dass der erste Share mit dem ersten Faktor und so weiter kombiniert wird.\par

Die Hashwerte der einzelnen Faktoren k�nnen gemeinsam in einer Datenbank gespeichert werden. Einzelne Shares entgegen d�rfen nie an dem gleichen Ort abgelegt sein, ansonsten besteht die Gefahr, dass ein Angreifer Zugriff auf zwei Shares gleichzeitig erlangt, welche ihn in dem hier verwendeten (2, 3)-Schwellenwertschema dazu erm�chtigen, das Geheimnis rekonstruieren zu k�nnen. Die empfohlene Aufteilung zur Speicherung der Shares ist wie folgt:

\begin{enumerate}
	\item \textit{Passwort-Share:} Der zum Passwort zugewiesene Share wird in einer Datenbank abgelegt.
	\item \textit{Biometrie-Share:} Der zum biometrischen Merkmal zugewiesene Share wird im sicheren Bereich des Smartphones abgelegt.
	\item \textit{Wiederherstellungsschl�ssel-Share:} Der zum Wiederherstellungsschl�ssel zugewiesene Share wird an einem vom Benutzer festgelegten (sicheren) Ort abgelegt (z. B. als QR-Code ausgedruckt auf einem Blatt Papier in einem Bankschlie�fach).
\end{enumerate}

Sobald alle Faktoren gehasht gespeichert und die Shares an den daf�r vorgesehenen Orten abgelegt worden sind, ist die Kontoerstellung vollst�ndig abgeschlossen. Die Best�tigung an den Nutzer schaltet gleichzeitig den vollen Zugriff auf die Webanwendung frei.

\subsubsection{Authentifizierung}
Eine erneute Authentifizierung ist notwendig, sobald der Nutzer auf der Webanwendung abgemeldet ist und erneut darauf zugreifen m�chte. F�r die Anmeldung werden lediglich zwei der drei Faktoren ben�tigt, welche der Benutzer bei Beginn des Login-Prozesses beliebig ausw�hlen kann.

\begin{itemize}
	\item \textit{Passwort}: Die Eingabe des Passworts ist nur �ber die Webanwendung m�glich. Die Freigabe des Shares erfolgt, wenn das eingegebene gehashte Passwort mit dem Hashwert aus der Datenbank �bereinstimmt.
	\item \textit{Biometrisches Merkmal}: In der Webanwendung wird der Benutzer darauf hingewiesen, sich bitte in der App mit dem biometrischen Merkmal (Finger- oder Gesichtsabdruck) zu verifizieren. Bei �bereinstimmung mit dem Wert aus der Datenbank wird der Share freigegeben.
	\item \textit{Wiederherstellungsschl�ssel}: Dieser kann sowohl �ber die Webanwendung als auch �ber die App eingegeben werden. Somit kann sich der Nutzer noch immer authentifizieren, auch wenn kein Zugriff auf das Smartphone m�glich ist. Die Eingabe per App bietet dar�ber hinaus allerdings noch die M�glichkeit, den QR-Code mittels der Kamera zu scannen, um eine schnellere Authentifizierung zu erm�glichen. Die Freigabe des Shares erfolgt nur, wenn der eingegebene gehashte Wiederherstellungsschl�ssel mit dem Hashwert aus der Datenbank �bereinstimmt.
\end{itemize}

Der Begriff \textquote{Freigabe} bedeutet in diesem Kontext, dass die Shares an eine zentrale Instanz \textquote{freigegeben} werden. Der Benutzer hat zu keinem Zeitpunkt Einsicht auf die konkreten Werte. Die erw�hnte Instanz k�nnte beispielsweise an eine extra abgesicherte Schnittstelle zwischen der Webanwendung/App und der Datenbank sein, an die die Shares �bertragen werden, welche sich dann um die Rekonstruktion des Geheimnisses k�mmert. Falls die Wiederherstellung des Geheimnisses m�glich ist und beide Hashwerte �bereinstimmen, wird der Benutzer authentifiziert. Andernfalls wird der Prozess abgebrochen und der Benutzer dar�ber informiert.

% Literaturverzeichnis
\printbibliography

\end{document}