% Praeamble
\documentclass[conference,10pt,a4paper]{IEEEtran}

\usepackage[T1]{fontenc}
\usepackage[ansinew]{inputenc}
\usepackage[ngerman]{babel}
\usepackage{lipsum} % Zum Generieren von Blindtext
\usepackage{csquotes} % Anf�hrungszeichen

\usepackage{amsmath}
\usepackage{pgfplots}
\pgfplotsset{compat=1.16}
\usepackage{graphicx}

\usepackage{hyperref}
\usepackage[figure]{hypcap}

\renewcommand\IEEEkeywordsname{Schl�sselw�rter}
\def\code#1{\texttt{#1}}

% Literaturverzeichnis
\usepackage[style=ieee,backend=biber]{biblatex}
\addbibresource{assets/literature.bib}
\renewcommand*{\bibfont}{\raggedright\small}

\begin{document}

% Angaben
\title{Analyse und Implementierung einer Multi-Faktor-Authentifizierung mit Shamir Secret Sharing}
\author{
	\IEEEauthorblockN{Nicolas Proske}
	\IEEEauthorblockA{
		\textit{Ostbayerische Technische Hochschule Amberg-Weiden} \\
		Moderne Anwendungen der Kryptographie \\
		E-Mail: \code{n.proske@oth-aw.de} \\
		Matr.-Nr.: \code{87672270}
	}
}

\maketitle

% Abstract
\begin{abstract}
\lipsum[1]
\end{abstract}

% Schluesselwoerter
\begin{IEEEkeywords}
MFA, Secret Sharing
\end{IEEEkeywords}

\section{Einleitung}
\lipsum[1] \autocite{sample-2023}

% Literaturverzeichnis
\printbibliography

\end{document}