% Praeamble
\documentclass[a4paper]{IEEEtran}

\usepackage[T1]{fontenc}
\usepackage[ansinew]{inputenc}
\usepackage[ngerman]{babel}

\usepackage{lipsum} % Zum Generieren von Blindtext
\usepackage{csquotes} % Anf�hrungszeichen

% Weitere Symbole
\usepackage{amsmath}
\usepackage{amssymb}

\usepackage{hyperref} % Klickbare Verlinkungen
\usepackage[figure]{hypcap}

\usepackage{listings}
\usepackage{xcolor}

 % Tabelle
\usepackage{tabularx}
\setlength\extrarowheight{4pt}
\usepackage{caption}

\definecolor{codegray}{rgb}{0.5, 0.5, 0.5}
\definecolor{darkgray}{rgb}{0.3, 0.3, 0.3}
\definecolor{backcolor}{RGB}{247, 247, 247}

\lstset{
    language=Python,
    basicstyle=\linespread{1.2}\ttfamily\footnotesize,
    keywordstyle=\color{blue}\ttfamily,
    stringstyle=\color{codegray}\ttfamily,
    commentstyle=\color{teal}\ttfamily,
    morecomment=[l][\color{teal}]{\#},
    backgroundcolor=\color{backcolor},
    rulecolor=\color{codegray},
    showspaces=false,
    showstringspaces=false,
    showtabs=false,
    frame=single,
    numbers=left,
    numberstyle=\scriptsize\color{codegray},
    breaklines=true,
    breakatwhitespace=true,
    captionpos=b
}

% Eingebetteter Code
\usepackage{xparse}
\NewDocumentCommand{\codeword}{v}{%
\texttt{\ttfamily\small\textcolor{darkgray}{#1}}%
}

% Diagramme
\usepackage{tikz}
\usetikzlibrary{shapes, arrows}
\tikzstyle{process} = [rectangle, draw, text centered, minimum height=1.5em]
\tikzstyle{decision} = [diamond, draw, text centered]
\tikzstyle{data}=[trapezium, draw, text centered, trapezium left angle=60, trapezium right angle=120, minimum height=1.5em]
\tikzstyle{connector} = [draw, -latex']

\renewcommand\IEEEkeywordsname{Schl�sselw�rter}
\def\code#1{\texttt{#1}}

% Literaturverzeichnis
\usepackage[style=ieee,backend=biber]{biblatex}
\addbibresource{assets/literature.bib}
\renewcommand*{\bibfont}{\raggedright\small}

\makeatletter
\newcommand*\titleheader[1]{\gdef\@titleheader{#1}}
\AtBeginDocument{%
  \let\st@red@title\@title
  \def\@title{%
    \bgroup\normalfont\large\centering\@titleheader\par\egroup
    \vskip1.5em\st@red@title}
}
\makeatother

% Angaben
\title{Analyse und Implementierung einer Multi-Faktor-Authentifizierung mit \\ Shamir's Secret Sharing}
\titleheader{Moderne Anwendungen der Kryptographie im Sommersemester 2023}
\author{
	\IEEEauthorblockN{Nicolas Proske} \\
	\IEEEauthorblockA{
		\textit{Ostbayerische Technische Hochschule Amberg-Weiden} \\
		E-Mail: \code{n.proske@oth-aw.de} \\
		Matr.-Nr.: \code{87672270}
	}
}

\begin{document}

\maketitle

% Abstract
\begin{abstract}
Heutige Authentifizierungsmethoden beruhen noch oft auf dem Ein-Faktor-Prinzip, wobei dieser eine Faktor, meist ein Passwort, die dahinter liegenden Daten nicht ausreichend vor unautorisiertem Zugriff sch�tzen kann. Im Laufe der Zeit hat sich deshalb die Zwei-Faktor-Authentifizierung etabliert, welche zus�tzlich ein weiteres Merkmal beim Login-Prozess voraussetzt. Diese Studienarbeit betrachtet dar�ber hinaus ein m�gliches Verfahren, wie ein vollumf�nglicher Login-Prozess mit mehreren Faktoren, verteilt auf unterschiedlichen Systemen, mit Hilfe von Multi-Faktor-Authentifizierung und Shamir's Secret Sharing realisiert werden kann.\par

Die Kombination aus einem Passwort, einem biometrischen Merkmal (zum Beispiel einem Fingerabdruck) und einem Wiederherstellungsschl�ssel erh�ht nicht nur die Sicherheit eines Systems, sondern dank dem Einsatz von Secret Sharing auch die Flexibilit�t des Benutzers, da nur zwei der drei Faktoren zur Authentifizierung ben�tigt werden.\par

Die Integration dieser Idee in eine Webanwendung in Verbindung mit einer dazugeh�rigen App zeigt eine sichere M�glichkeit auf, wie ein Login-Prozess in einer modernen Anwendung ablaufen kann.
\end{abstract}

% Schluesselwoerter
\begin{IEEEkeywords}
Authentifizierung, Secret Sharing, Web, App
\end{IEEEkeywords}

\section{Einleitung}
\lipsum[1] \autocite{sample-2023}
\section{Authentifizierung mit Faktoren}
\subsection{Ein-Faktor-Authentifizierung}
Lange Zeit haben Authentifizierungsmethoden auf einem einzelnen Identifikationsfaktor beruht, in der Regel einer Kombination aus Benutzername und Passwort. Wenn diese beiden Parameter �ber mehrere Dienste hinweg identisch sind, bedeutet dies, dass ein Angreifer, der ein einziges Konto kompromittiert, automatisch Zugriff auf die anderen Konten erh�lt \textemdash{} Dabei spielt die St�rke des Passworts keine Rolle. Dieser One-Factor-Authentication (OFA) Ansatz  hat jahrzehntelang das R�ckgrat der Informationssicherheit gebildet. Dennoch hat sich angesichts der zunehmenden Komplexit�t von Cyberbedrohungen gezeigt, dass die Abh�ngigkeit von einem einzigen Faktor f�r die Authentifizierung eine Schwachstelle darstellt, die anf�llig f�r verschiedene Verletzungen wie Brute-Force-Angriffe, Phishing und Social Engineering ist. Diese Schwachstellen verdeutlichen, dass OFA f�r heutige Anwendungsf�lle in aller Regel keine ausreichende Sicherheit mehr bietet.

\subsection{Multi-Faktor-Authentifizierung}
Multi-Factor-Authentication (MFA) stellt einen signifikanten Fortschritt in der Evolution der digitalen Sicherheitsma�nahmen dar. Im Gegensatz zur Einzelfaktor-Authentifizierung, die �blicherweise auf einer einzigen Form des Nachweises wie einem Passwort basiert, erh�ht MFA die Sicherheit durch zus�tzliche Schutzebenen, indem mehrere unabh�ngige Zugangsdaten f�r die Authentifizierung erforderlich sind. Diese Zugangsdaten k�nnen in drei Hauptkategorien eingeteilt werden:

\begin{enumerate}
	\item \textit{Wissen}: Informationen, die der Benutzer kennt, wie Passw�rter, PINs und Antworten auf geheime Fragen.
  	\item \textit{Eigenheit}: Biologische Merkmale, die einzigartig f�r den Benutzer sind, wie Fingerabdr�cke, Netzhautmuster oder Gesichtserkennung.
  	\item \textit{Besitz}: Gegenst�nde oder Ger�te, die der Benutzer besitzt, wie Smartphones, Chipkarten oder physische Schl�ssel. Die Best�tigung des Besitzes kann verschiedene Formen annehmen, angefangen von der Entgegennahme und Eingabe eines per SMS an eine registrierte Telefonnummer gesendeten Codes bis hin zum Einsetzen eines physischen Schl�ssels in ein Schloss.
\end{enumerate}

Der Hauptvorteil von MFA gegen�ber OFA liegt daher im schichtbasierten Ansatz. Selbst wenn ein Angreifer es schafft, einen Authentifizierungsfaktor zu umgehen, bieten die verbleibenden Faktoren weiterhin Schutz. Eine Kompromittierung eines Faktors gef�hrdet also nicht die Gesamtsicherheit. Trotz der St�rken bringt eine MFA auch eigene Herausforderungen mit sich, wie beispielsweise die potenziell erh�hte Komplexit�t und die Notwendigkeit f�r Benutzer, mehrere Authentifizierungsfaktoren zu verwalten. Dennoch �berwiegen die Vorteile der MFA oft diese potenziellen Nachteile, insbesondere in Umgebungen, in denen der Schutz sensibler Daten oberste Priorit�t hat.
\section{Shamir's Secret Sharing}
Secret Sharing ist ein grundlegender Baustein der modernen Kryptographie. Eines der bekanntesten Verfahren wurde am 1. November 1979 ver�ffentlicht und ist nach seinem Erfinder Adi Shamir, einem israelischen Kryptographen, benannt: Shamir's Secret Sharing \autocite{shamir-secretsharing-1979}.\par

Es basiert auf der Idee, ein Geheimnis in mehrere Teile, sogenannte Shares, aufzuteilen. Um das Geheimnis wiederherzustellen, m�ssen eine bestimmte Anzahl dieser Shares zusammengebracht werden. Jeder einzelne Share ist f�r sich genommen bedeutungslos und gibt keinerlei Informationen preis. Ein Schwellenwert definiert die minimale Anzahl von Shares, die erforderlich sind, um das Geheimnis wiederherstellen zu k�nnen. Dies stellt sicher, dass das Geheimnis selbst dann sicher bleibt, wenn ein Teil der Shares verloren gehen oder in die H�nde eines Angreifers gelangen. Das dabei verwendete (k, n)-Schwellenwertschema legt fest, wie viele $k$ Shares ben�tigt werden, um auf das Geheimnis zu kommen, $n$ ist gr��er $k$ und bezieht sich auf die Gesamtzahl der Shares, in die das Geheimnis aufgeteilt wird.

\subsection{Mathematische Veranschaulichung}
Shamir's Secret Sharing basiert auf dem Prinzip der Polynominterpolation in endlichen K�rpern, wobei $k$ Punkte ein Polynom vom Grad $k - 1$ eindeutig definieren. Um dies anhand eines mathematischen Beispiels zu veranschaulichen, wird im Folgenden ein (2, 3)-Schwellenschema mit $k = 2$ und $n = 3$ betrachtet, bei dem das Geheimnis $S$ der Zahl $42$ entspricht. Sei $p = 43$ eine Primzahl mit $p > S$. Alle Berechnungen erfolgen im endlichen K�rper $\mathbb{F}_p$.

\subsubsection{Generierung der Shares}
Der erste Schritt besteht darin, ein Polynom vom Grad $k - 1 = 2 - 1 = 1$ aufzustellen:

\begin{equation*}
f(x) = mx + b \mod{p}
\end{equation*}

Die Konstante $b$ entspricht dabei dem Geheimnis $S$. Aus Gr�nden der �bersichtlichkeit wird in diesem Beispiel $m = 4$ gew�hlt:

\begin{equation*}
f(x) = 4x + 42 \mod{43}
\end{equation*}

Im n�chsten Schritt erfolgt die Berechnung von $n$ Punkten in der Ebene. Dazu wird f�r $x = 1...n$ eingesetzt:

\begin{equation*}
\begin{aligned}
&\text{F�r } x = 1: y_1 = f(1) = 4*1 + 42 \mod{43} = 3 \\
&\text{F�r } x = 2: y_2 = f(2) = 4*2 + 42 \mod{43} = 7 \\
&\text{F�r } x = 3: y_3 = f(3) = 4*3 + 42 \mod{43} = 11
\end{aligned}
\end{equation*}

Jeder der berechneten Punkte $(1, 3), (2, 7), (3, 11)$ repr�sentiert dabei einen Share. \autoref{tab:shares} zeigt die Verteilung aller drei Shares an insgesamt drei unterschiedliche Nutzer:

\begin{table}[!h]
\centering
\normalsize
\begin{tabularx}{0.45\textwidth}{*3{>{\centering\arraybackslash}X}@{}}
\textit{Verteilung an} user($i$) & \textit{$f(i)$} & share($i, f(i) \mod{p}$) \\
\hline
1 & $f(1) = 46$ & (1, 3) \\
2 & $f(2) = 50$ & (2, 7) \\
3 & $f(3) = 54$ & (3, 11) \\
\end{tabularx}
\medskip
\caption{Verteilung der Shares}
\label{tab:shares}
\end{table}

\subsubsection{Rekonstruktion mit linearem Gleichungssystem}
Sind nun $k$ Punkte gegeben, kann das urspr�ngliche Geheimnis rekonstruiert werden. Im Folgenden werden die Punkte $(1, 3)$ und $(2, 7)$ als Gleichungen in einem linearen Gleichungssystem dargestellt:

\begin{equation*}
\begin{aligned}
&\text{1. } m + b = 3 \\
&\text{2. } 2m + b = 7
\end{aligned}
\end{equation*}

Die Unbekannten werden nun �ber das Substitutionsverfahren gel�st. Durch Umstellen der ersten Gleichung nach $b$ folgt $b = 3 - m$. Dieser Ausdruck wird in die zweite Gleichung eingesetzt, was zu $2m + (3 - m) = 7$ f�hrt. Daraus folgt $m = 4$. Die erhaltene L�sung f�r $m$ wird dann in die umgestellte erste Gleichung eingesetzt, um $b$ zu berechnen: $b = 3 - 4 = -1$. Da die Berechnungen im endlichen K�rper $\mathbb{F}_{43}$ durchgef�hrt werden, wird das Ergebnis $\mod{43}$ genommen, um das Geheimnis im Wertebereich von $0$ bis $p-1$ zu erhalten: $b = -1 \mod{43} = 42$, was dem Geheimnis $S = 42$ entspricht. Bei gr��eren Werten von $k$ w�rde ein Polynom h�heren Grades und ein entsprechend gr��eres lineares Gleichungssystem entstehen.

\subsubsection{Rekonstruktion mit Lagrange-Interpolations-Formel}
Die Lagrange-Interpolation ist das in der Praxis am h�ufigsten eingesetzte Verfahren zur Bestimmung des Polynoms einer bestimmten Ordnung, das durch eine gegebene Menge von Punkten verl�uft. Diese Methode bietet den Vorteil, dass sie direkt eine Formel zur Rekonstruktion des Geheimnisses liefert, ohne dass ein Gleichungssystem explizit gel�st werden muss. Die Koeffizienten $m_0, ..., m_{k-1}$ eines unbekannten Polynoms $f$ vom Grad $k-1$ aus $k$ Punkten $(x_i, y_i)$ k�nnen wie folgt berechnet werden:

\[
f(x) = \sum_{i=1}^{k} \left[ y_i \cdot \prod_{\substack{1 \leq j \leq k \\ i \neq j}} \frac{x - x_j}{x_i - x_j} \right] \mod{p}
\]

Unter Verwendung dieser Formel l�sst sich $m_0 = f(0)$ und damit das Geheimnis $S$ aus $k$ gegebenen Punkten berechnen \autocite[S. 65 f.]{bsi-richtlinie-2023}.
\section{Die Idee der Kombination}
Das Ziel dieser Arbeit besteht darin, die Vorteile beider Verfahren zu kombinieren, um eine robuste und sichere Authentifizierungsmethode zu entwickeln. Im Kern wird ein Geheimnis aus mehreren Authentifizierungsfaktoren generiert und mithilfe von Shamir's Secret Sharing aufgeteilt. Die zur Multi-Faktor-Authentifizierung verwendeten Merkmale entsprechen einem Passwort (Wissen), dem Fingerabdruck des Nutzers (Eigenheit) und einem Wiederherstellungsschl�ssel in Form eines QR-Codes (Besitz). Die Kombination aller drei Faktoren dient als Grundlage zur Berechnung des Geheimnisses, welches anschlie�end mittels Shamir's Secret Sharing in drei Shares aufgeteilt wird, wobei nur zwei St�ck zur sp�teren Authentifizierung ben�tigt werden. Jeder Share wird anschlie�end eindeutig zu einem der obigen Faktoren zugeordnet und auf unterschiedlichen Wegen sicher abgelegt, zum Beispiel der Passwort-Share in einer Datenbank, der Fingerabdruck-Share im sicheren Bereich eines Smartphones und der zugeh�rige Share zum Wiederherstellungsschl�ssel als QR-Code ausgedruckt an einem sicheren Ort.

\subsection{Vorteile dieses Konzepts}
\subsubsection{Erh�hte Sicherheit}
MFA und SSS erg�nzen sich gegenseitig, um eine robuste Sicherheitsarchitektur zu schaffen. W�hrend MFA bereits eine zus�tzliche Sicherheitsebene durch die Verwendung mehrerer Faktoren bietet, stellt Shamir's Secret Sharing sicher, dass die zu sch�tzenden Daten unverschl�sselt dezentral gespeichert werden k�nnen, da ein einzelner Share keine R�ckschl�sse auf das Geheimnis zul�sst. Dadurch wird das Risiko eines vollst�ndigen Datenlecks oder unbefugten Zugriffs drastisch minimiert.

\subsubsection{Flexibilit�t}
Benutzer k�nnen aus drei verschiedenen Optionen (Passwort, Fingerabdruck und Wiederherstellungsschl�ssel) zur Authentifizierung w�hlen. Dar�ber hinaus kann die Aufteilung von Shares an unterschiedliche Ger�te oder Personen erfolgen, um den Bed�rfnissen und Anforderungen eines Nutzers gerecht zu werden.

\subsubsection{Schutz vor Datenverlust}
Wenn beispielsweise ein Benutzer sein Smartphone verliert oder es ir�re�pa�ra�bel besch�digt wurde, ist weiterhin ein Zugriff �ber die beiden anderen Faktoren gew�hrleistet. Dies bietet einen zus�tzlichen Schutz vor Datenverlust.
\section{Implementierung}
Das folgende Kapitel beschreibt schrittweise die praktische Umsetzung einer in Python (Version 3.10.9) programmierten Kombination von Multi-Faktor-Authentifizierung und Shamir's Secret Sharing. Dieser Prozess gliedert sich in drei Phasen: \textit{Konstruktion des Geheimnisses}, \textit{Generierung der Shares} und \textit{Authentifizierung}. In der ersten Phase wird eine nat�rliche Zahl auf Grundlage von drei verschiedenen Faktoren konstruiert. Diese dient in Phase 2 als Geheimnis f�r die Anwendung von Shamir's Secret Sharing, um daraus drei Shares zu erzeugen. F�r die abschlie�ende Authentifizierung in Phase 3 werden zwei dieser Shares ben�tigt.\par

\textit{Anmerkung:} Die nachfolgend pr�sentierten Code-Abschnitte dienen haupts�chlich der Veranschaulichung und sind ohne zus�tzliche Anpassungen und Erg�nzungen nicht zwingend lauff�hig.

\subsection{Phase 1: Konstruktion des Geheimnisses}
Vor der Durchf�hrung einer Authentifizierung ist es notwendig, die daf�r ben�tigten Shares zu generieren. Als Grundlage dient hierbei ein Geheimnis, das in diesem Fall auf Basis von drei verschiedenen Authentifizierungsfaktoren erzeugt wird.

\subsubsection{Nutzer legt drei Faktoren fest}
Bei den Faktoren handelt es sich um ein Passwort, einen Fingerabdruck und einen Wiederherstellungsschl�ssel. Wie in \autoref{code:input_factors} gezeigt, werden die ersten beiden Faktoren vom Nutzer bereitgestellt, w�hrend der dritte Faktor zuf�llig in Form eines 128 Bit langen hexadezimalen Strings generiert wird.

\begin{lstlisting}[caption={Initialisierung der drei Faktoren},label=code:input_factors,numbers=left]
password = input() # 1. Faktor
fingerprint = input() # 2. Faktor
recovery_key = os.urandom(16).hex() # 3. Faktor
\end{lstlisting}

\subsubsection{Faktoren umwandeln}
All diese Faktoren werden im sp�teren Verlauf f�r die Authentifizierung ben�tigt. Daher ist es wichtig, dass diese Informationen umgewandelt werden, um m�gliche R�ckschl�sse auf die urspr�nglichen Eingaben des Nutzers auszuschlie�en. Aus diesem Grund werden alle Faktoren in einem weiteren Schritt mittels der SHA-256-Hashfunktion umgewandelt (siehe \autoref{fig:hash_factors}).

\begin{figure}[!h]
\centering
\begin{tikzpicture}

\node [data, fill=orange!10] at (-2.5,0) (password) {Passwort};
\node [data, fill=blue!10] at (-0,0) (fingerprint) {Fingerabdruck};
\node [data, fill=green!10] at (2.5,0) (recovery_key) {Schl�ssel};
\node [data, fill=gray!10] at (0,-1.25) (hash_factors) {SHA-256-Hash};
\node [data, fill=orange!10] at (-2.5,-2.25) (password_hash) {807A09...};
\node [data, fill=blue!10] at (-0,-3) (fingerprint_hash) {8587EC...};
\node [data, fill=green!10] at (2.5,-2.25) (recovery_key_hash) {39B997...};

\node[draw=none] at (-3.25, -0.75) {W@6a...};
\node[draw=none] at (0.6, -0.625) {1011...};
\node[draw=none] at (3.1, -0.75) {cf61...};

\path [connector] (password) |- (hash_factors);
\path [connector] (fingerprint) -- (hash_factors);
\path [connector] (recovery_key) |- (hash_factors);
\path [connector] (hash_factors) |- (password_hash);
\path [connector] (hash_factors) -- (fingerprint_hash);
\path [connector] (hash_factors) |- (recovery_key_hash);

\end{tikzpicture}
\caption{Faktoren umwandeln}
\label{fig:hash_factors}
\end{figure}

Zur Realisierung im Quelltext nimmt die in \autoref{code:hash_factors} gegebene Funktion \codeword{hash_string} einen String \codeword{value} entgegen. Dieser Wert wird zun�chst mit \codeword{.encode()} als Bytes repr�sentiert und unter Verwendung der \codeword{hashlib}-Bibliothek in einen SHA-256-Hash konvertiert. Nach Anwendung der Funktion auf die vom Nutzer eingegebenen Faktoren wird die \codeword{.digest()}-Methode auf den berechneten Hash angewendet, um das Ergebnis als Bytefolge zur�ckzugeben, um diese im nachfolgenden Schritt in eine ganze Zahl umwandeln zu k�nnen.

\begin{lstlisting}[caption={Hashen der drei Faktoren},label=code:hash_factors,numbers=left]
def hash_string(value):
    return hashlib.SHA-256(value.encode())

password_hash = hash_string(password).digest()
fingerprint_hash = hash_string(fingerprint).digest()
recovery_key_hash = hash_string(recovery_key).digest()
\end{lstlisting}

\subsubsection{Interpretation der Hashwerte als Zahlen}
Alle drei erhaltenen Hashes m�ssen nun als Zahlen interpretiert werden, da Shamir's Secret Sharing eine ganze Zahl f�r das Geheimnis fordert. Die Funktion \codeword{hash_to_int} aus \autoref{code:interpret_hashes} nimmt ebenfalls einen Parameter \codeword{value} entgegen, der hier die zuvor generierte Bytefolge darstellt. Durch die Verwendung der Methode \codeword{int.from_bytes()} mit dem Parameter \codeword{value} wandelt die Funktion diese Bytefolge in eine Ganzzahl um. Dabei erfolgt die Interpretation der Bytes in der Reihenfolge \textquote{big}, wodurch das Most Significant Bit zuerst und das Least Significant Bit zuletzt ber�cksichtigt wird. Das Ergebnis der Funktion, eine Ganzzahl, wird zur�ckgegeben. Anschlie�end wird diese Funktion auf die Hashwerte von Passwort, Fingerabdruck und Wiederherstellungsschl�ssel angewendet und die Zahlen in den entsprechenden Variablen gespeichert.

\begin{lstlisting}[caption={Hashwerte als Zahlen interpretieren},label=code:interpret_hashes,numbers=right]
def hash_to_int(value):
    return int.from_bytes(value, byteorder="big")

password_number = hash_to_int(password_hash)
fingerprint_number = hash_to_int(fingerprint_hash)
recovery_key_number = hash_to_int(recovery_key_hash)
\end{lstlisting}

\subsubsection{Geheimnis erzeugen}
Das Geheimnis ergibt sich nun durch die Aneinanderreihung aller Zahlen. Hierbei werden die Zahlen nicht addiert, sondern in zuf�lliger Reihenfolge konkateniert. In \autoref{code:concat_numbers} wird zuerst eine Liste \codeword{numbers} erstellt, die die Ganzzahlen aus \autoref{code:interpret_hashes} enth�lt. Anschlie�end wird die Liste durch Anwendung der \codeword{shuffle}-Methode aus der Bibliothek \textquote{random} zuf�llig durchmischt. Die so neu geordneten Zahlen werden in einer Schleife durchlaufen, jeder Wert in einen String umgewandelt und an den vorherigen Wert angehangen. Dieser zusammengesetzte String wird schlussendlich in einen Integer umgewandelt und als tempor�re Variable \codeword{S} zwischengespeichert (siehe \autoref{code:concat_numbers}).

\begin{lstlisting}[caption={Zahlen zu Geheimnis konkatenieren},label=code:concat_numbers,numbers=right]
numbers = [password_number, fingerprint_number, recovery_key_number]
random.shuffle(numbers)
S = int("".join(str(num) for num in numbers))
\end{lstlisting}

Um das Geheimnis w�hrend der Authentifizierung bei einer erfolgreichen Rekonstruktion auf �bereinstimmung pr�fen zu k�nnen, muss es sp�ter abrufbar sein. Dazu wird es in \autoref{code:hash_secret} mit SHA-256 gehasht. Dadurch wird sichergestellt, dass das zu sch�tzende Geheimnis f�r sp�tere Zwecke ohne Bedenken in einer Datenbank gespeichert werden kann.

\begin{lstlisting}[caption={Geheimnis hashen},label=code:hash_secret,numbers=right]
S_hash = hash_string(str(S)).hexdigest()
\end{lstlisting}

\subsection{Phase 2: Generierung der Shares}
Das originale, nicht gehashte Geheimnis \codeword{S} wird in der zweiten Phase dazu ben�tigt, um es mit Hilfe von Shamir's Secret Sharing in einzelne Shares zu zerlegen.

\subsubsection{Primzahl erzeugen}
Alle Berechnungen erfolgen wie auch zu Beginn in der mathematischen Veranschaulichung in einem endlichen K�rper. Dieser wird definiert als $\mathbb{F}_p$, wobei $p$ eine Primzahl gr��er $n$ und $S$ ist. Die Bibliothek \textquote{libnum} stellt die Funktion \textquote{generate\_prime} bereit, die unter Eingabe einer Bitl�nge die Primzahl in dieser Gr��enordnung erzeugt. Zur Erzeugung einer solchen Primzahl wird zun�chst die Bitl�nge von $S$ ermittelt und mit der Zahl $2$ multipliziert, um sicherzustellen, dass die generierte Primzahl der durch das BSI vorgegebenen Bedingung $p \geq max(2*r, n + 1)$, wobei $r$ die Bitl�nge des Geheimnisses $S$ repr�sentiert, entspricht \autocite[S. 66]{bsi-richtlinie-2023}. Diese Primzahl aus \autoref{code:gen_prime} f�r die nachfolgenden Berechnungen verwendet.

\begin{lstlisting}[caption={Primzahl erzeugen},label=code:gen_prime,numbers=left]
bit_length = max(2 * libnum.len_in_bits(S), 4)
p = libnum.generate_prime(bit_length)

assert p > S, "Primzahl kleiner als Geheimnis"
\end{lstlisting}

\subsubsection{Shares erzeugen}
Nachdem alle Vorbereitungen abgeschlossen sind, wird im letzten Schritt das Geheimnis in einzelne Shares zerlegt. Das verwendete (2, 3)-Schwellenwertschema erzeugt insgesamt drei Shares, wovon zwei zur Rekonstruktion ben�tigt werden. Die Funktion \codeword{create_shares(S, p)} in \autoref{code:create_shares} generiert eine Liste von Punkten, die f�r die Rekonstruktion des Geheimnisses im sp�teren Verlauf ben�tigt werden. Zu Beginn wird ein Koeffizient $m$ erzeugt, der als Ganzzahl aus 32 zuf�lligen Bytes interpretiert wird. Dieser Koeffizient wird im n�chsten Schritt dazu verwendet, die y-Koordinaten der Punkte zu berechnen. Daf�r wird �ber alle x-Werte von 1 bis einschlie�lich 3 iteriert und der dazugeh�rige y-Wert �ber $(m * x + S) \mod{p}$ berechnet, wobei $S$ das Geheimnis und $p$ die errechnete Primzahl ist. Jeder berechnete Punkt $(x_i, y_i)$ wird zur Liste \codeword{shares} hinzugef�gt. Am Ende wird diese Liste, die die generierten Punkte beziehungsweise Shares enth�lt, zur�ckgegeben.

\begin{lstlisting}[caption={Shares erzeugen},label=code:create_shares,numbers=left]
def create_shares(S, p):
    m = int.from_bytes(os.urandom(32), byteorder="big") # Pseudozuf�lliger Koeffizient
    
    shares = []
    
    for x in range(1, 4): # x aufsteigend iterieren
        y = (m * x + S) % p # y-Wert berechnen
        shares.append((x, y)) # Punkt hinzuf�gen
    return shares

shares = create_shares(S, p)
\end{lstlisting}

\subsection{Phase 3: Authentifizierung}
Um das Geheimnis wiederherzustellen und die Authentifizierung durchzuf�hren, m�ssen zwei Faktoren durch den Benutzer angegeben werden. Nach der vollst�ndigen Eingabe erfolgt die Authentifizierung.

\subsubsection{Auswahl der Faktoren}
In diesem Schritt wird der Benutzer zur Eingabe der gew�nschten Faktoren aufgefordert. Der Nutzer gibt zwei Zahlen, getrennt durch ein Leerzeichen, ein. Jede Zahl steht f�r einen der drei m�glichen Faktoren: Passwort (1), Fingerabdruck (2) oder Wiederherstellungsschl�ssel (3). Der Benutzer muss dabei zwei unterschiedliche Zahlen ausw�hlen und jede dieser Zahlen muss entweder 1, 2 oder 3 entsprechen.\par

\autoref{code:create_shares} zeigt die Implementierung. Zuerst wird der Benutzer dazu aufgefordert, seine gew�nschten Faktoren �ber die Tastatur einzugeben. Die Eingabe muss zwei Zeichen enthalten, die durch ein Leerzeichen getrennt sind. Jedes durch ein Leerzeichen getrenntes Zeichen wird als separates Element in einer Liste gespeichert. Als n�chstes wird die Funktion \codeword{map()} verwendet, um alle Elemente in der Liste in ganze Zahlen umzuwandeln. Um die Zahlen nun in einer Liste zu speichern, wird das \codeword{map}-Objekt mit der Funktion \codeword{list()} als Liste ausgegeben. Schlie�lich wird mit \codeword{[:2]} der Slicing-Operator angewendet, um sicherzustellen, dass nur die ersten beiden Elemente der Liste, also die zwei vom Benutzer eingegebenen Zahlen, ber�cksichtigt werden. Selbst wenn der Benutzer mehr als zwei Zahlen eingibt, werden nur die ersten beiden Zahlen f�r weitere Verarbeitungsschritte verwendet.\par

Nachdem der Benutzer seine Eingabe get�tigt hat, wird �berpr�ft, ob die Eingabe korrekt ist und den erforderlichen Kriterien entspricht. Verschiedene \codeword{assert}-Anweisungen werden verwendet, um die Pr�fung durchzuf�hren und im Falle von Fehlern entsprechende Fehlermeldungen auszugeben:

\begin{itemize}
	\item Die erste \codeword{assert}-Anweisung stellt sicher, dass der Benutzer genau zwei Zahlen eingegeben hat. Wenn die Anzahl der eingegebenen Zahlen nicht genau zwei betr�gt, wird eine Fehlermeldung mit dem Text \textquote{Bitte verwende genau 2 Zahlen} angezeigt.
	\item Die zweite \codeword{assert}-Anweisung �berpr�ft, ob alle eingegebenen Zeichen entweder 1, 2 oder 3 entsprechen. Falls eine der eingegebenen Zahlen nicht zu den erlaubten Werten geh�rt, wird eine Fehlermeldung mit dem Text \textquote{Bitte verwende nur 1, 2 oder 3} ausgegeben.
	\item Die dritte \codeword{assert}-Anweisung garantiert, dass der Benutzer zwei unterschiedliche Zahlen eingegeben hat. Wenn die beiden eingegebenen Zahlen identisch sind, wird eine Fehlermeldung mit dem Text \textquote{Bitte verwende zwei unterschiedliche Zahlen} angezeigt.
\end{itemize}

\begin{lstlisting}[caption={Faktoren ausw�hlen},label=code:choose_factors,numbers=right]
factors = list(map(int, input().split()))[:2]

assert len(factors) == 2
assert all(factor in [1,2,3] for factor in factors)
assert factors[0] != factors[1]
\end{lstlisting}

\subsubsection{Eingabe der Faktoren}
In diesem Abschnitt liegt der Fokus auf der Aufforderung an den Benutzer, die ausgew�hlten Authentifizierungsfaktoren einzugeben. \autoref{code:map_factors} zeigt die Variable \codeword{num_factors_map}, ein W�rterbuch, dass den Zahlen 1, 2 und 3 die entsprechenden Werte f�r Passwort, Fingerabdruck und Wiederherstellungsschl�ssel zuweist. Dieses W�rterbuch wird sp�ter verwendet, um die vom Benutzer eingegebenen Faktoren mit den gespeicherten Werten abzugleichen. 

\begin{lstlisting}[caption={Zahlen zu Faktoren zuweisen},label=code:map_factors,numbers=left]
num_factors_map = {
    1: password,
    2: fingerprint,
    3: recovery_key
}
\end{lstlisting}

Die in \autoref{code:input_new_factors} definierte for-Schleife erm�glicht es, �ber die Liste der vom Benutzer ausgew�hlten Faktoren zu iterieren. Je nachdem, welche Zahlen der Nutzer im letzten Schritt gew�hlt hat, wird er nun dazu aufgefordet, den dazugeh�rigen Wert einzugeben. Falls die Faktoren beispielsweise den Werten 1 und 3 entsprechen, muss der Benutzer nacheinander sein zu Beginn festgelegtes Passwort (1) und den Wiederherstellungsschl�ssel (3) eingeben. Anschlie�end erfolgt eine �berpr�fung mit \codeword{if user_input == num_factors_map[i]}, um festzustellen, ob die Eingabe des Benutzers mit dem gespeicherten Wert des entsprechenden Authentifizierungsfaktors �bereinstimmt. Ist dies der Fall, wird der zugeh�rige Share zur Liste \codeword{user_shares} hinzugef�gt. Wenn die Eingabe nicht �bereinstimmt, wird eine Fehlermeldung angezeigt und der Authentifizierungsprozess abgebrochen.

\begin{lstlisting}[caption={Eingabe der Faktoren},label=code:input_new_factors,numbers=left]
user_shares = []

for i in factors:
    if i == 1:
        print("\nBitte Passwort eingeben:")
    elif i == 2:
        print("\nBitte Fingerabdruck eingeben:")
    elif i == 3:
        print("\nBitte Wiederherstellungsschl�ssel eingeben:")

    user_input = input()

    # Pr�fung auf �bereinstimmung
    if user_input == num_factors_map[i]:
        # Eingabe stimmt mit urspr�nglichem Wert �berein, dazugeh�riges Share zu Liste hinzuf�gen
        user_shares.append(shares[i - 1])
\end{lstlisting}

\subsubsection{Geheimnis rekonstruieren}
Mit den erhaltenen Shares kann das Geheimnis nun rekonstruiert werden.\par

Die Funktion \codeword{lagrange} in \autoref{code:lagrange_interpolation} berechnet den spezifischen Lagrange-Koeffizienten f�r den gegebenen Punkt $i$ in der Liste der x-Werte \codeword{x_values} im K�rper $\mathbb{F}_p$.\par

Die innere Schleife durchl�uft alle x-Werte in \codeword{x_values} und multipliziert das bisherige Ergebnis mit dem Kehrwert des Differenzterms $x_i - x_j$. Diese Kehrwerte werden modulo $p$ berechnet, um im endlichen K�rper zu bleiben. Hierbei wird \codeword{pow((x_values[i] - x_values[j]), p-2, p)} verwendet, um das multiplikative Inverse unter Modulo $p$ zu berechnen (basierend auf dem kleinen Satz von Fermat). Um den Wert des Lagrange-Koeffizienten f�r $x = 0$ zu berechnen, wird in \autoref{code:lagrange_interpolation} \codeword{0 - x_values[j]} geschrieben.

Wenn alle Multiplikationen durchgef�hrt sind, steht in \codeword{result} der Lagrange-Koeffizient f�r den Punkt $i$. Diese Koeffizienten werden dann verwendet, um eine Polynomfunktion zu erstellen, die zur Wiederherstellung des urspr�nglichen Geheimnisses verwendet wird.\par

\begin{lstlisting}[caption={Lagrange-Interpolation},label=code:lagrange_interpolation,numbers=right]
def lagrange(i, x_values):
    result = 1
    for j in range(len(x_values)):
        if j != i:
            result = (result * (0 - x_values[j]) * pow((x_values[i] - x_values[j]), p-2, p)) % p
    return result
\end{lstlisting}

Die in \autoref{code:reconstruct_secret} definierte Funktion \codeword{reconstruct_shares(shares, p)} nimmt als Argumente eine Liste von Shares und eine Primzahl zur Berechnung des Geheimnisses entgegen. Zun�chst werden die x-Werte aus den Punkten extrahiert, welche zur Berechnung der Lagrange-Koeffizienten verwendet werden. Die Funktion durchl�uft nun alle Shares. F�r jeden Share wird das Produkt aus dem y-Wert des jeweiligen Shares und dessen Lagrange-Koeffizienten berechnet und zu dem Geheimnis addiert. Abschlie�end wird das Geheimnis von der Funktion zur�ckgegeben. Der R�ckgabewert entspricht dem konstanten Term des rekonstruierten Polynoms und somit dem urspr�nglich geteilten Geheimnis.

\begin{lstlisting}[caption={Geheimnis rekonstruieren},label=code:reconstruct_secret,numbers=right]
def reconstruct_secret(shares, p):
    x_values = [share[0] for share in shares]
    secret = 0
    
    for i in range(len(shares)):
        secret = (secret + shares[i][1] * lagrange(i, x_values)) % p

    return secret
    
reconstructed_S = reconstruct_secret(user_shares, p)
\end{lstlisting}

\subsubsection{Pr�fung auf Korrektheit}
Auf das rekonstruierte Geheimnis wird im letzten Schritt die Hashfunktion SHA-256 angewendet, um es mit dem Hashwert des urspr�nglichen Geheimnisses vergleichen zu k�nnen. In \autoref{code:check_hashes} erfolgt im if-else-Block die Pr�fung, ob beide ermittelten Hashwerte �bereinstimmen. Wenn dies der Fall ist, bedeutet das, dass die Authentifizierung erfolgreich war.

\begin{lstlisting}[caption={Hashwerte �berpr�fen},label=code:check_hashes,numbers=left]
reconstructed_S_hash = hash_string(str(reconstructed_S)).hexdigest()

if reconstructed_S_hash == S_hash:
    success_print("Authentifizierung erfolgreich.")
else:
    raise Exception("Rekonstruktion nicht m�glich.")
\end{lstlisting}
\section{Relevante Sicherheitsaspekte}
Die Implementierung dient dazu, die Idee hinter der Kombination von MFA und SSS anschaulich zu vermitteln. Aus diesem Grund werden dort bestimmte Sicherheitsaspekte au�en vor gelassen, um den Quelltext �bersichtlich und leicht verst�ndlich zu halten. Im Folgenden wird daher auf relevante Kriterien eingegangen, die in einer Realisierung beachtet werden sollten.

\subsection{Randomisierung zur Geheimniserzeugung}
Das hier implementierte Geheimnis setzt sich einem Passwort, einem Fingerabdruck und einem zuf�llig generierten Wiederherstellungeschl�ssel zusammen. Die darauf berechneten Hashwerte werden als Zahlen interpretiert und zuf�llig aneinandergereiht, wodurch sich das Geheimnis ergibt. Dieser Ansatz ist nur sicher, solange ein Angreifer keinen Zugriff auf alle Hashwerte (oder die Faktoren selbst oder eine Mischung aus beidem) hat. Da dies nur in der Theorie immer der Fall ist, m�ssen zus�tzliche Sicherheitsebenen geschaffen werden, um das Geheimnis zu sch�tzen. Ist ein Angreifer in Besitz aller Hashwerte, kann das Geheimnis so in wenigen Schritten rekonstruiert werden, da nach dem Prinzip von Kerckhoffs die Sicherheit eines Verfahren von der Geheimhaltung der Schl�ssel, hier der Shares, abh�ngt und nicht von der Geheimhaltung des Algorithmus \textemdash{} Ein Angreifer kennt daher den Algorithmus und somit das Vorgehen zur Berechnung des Geheimnisses.\par

Die Randomisierung ist im Beispiel dieser Implementierung in Form der zuf�lligen Aneinanderreihung der als Zahlen interpretierten Hashwerte angedeutet. Durch Ausprobieren ben�tigt ein Angreifer bei drei Zahlen im schlechtesten Fall jedoch nur sechs Versuche, um alle m�glichen Kombinationen auszuprobieren und das Geheimnis zu erhalten. F�r die erste Zahl gibt es $n$ M�glichkeiten, f�r die zweite Zahl (da die erste Zahl bereits ausgew�hlt wurde) $n-1$ M�glichkeiten und f�r die dritte Zahl (da bereits zwei Zahlen ausgew�hlt wurden) $n-2$ M�glichkeiten. Die Anzahl der M�glichkeiten f�r die drei gegebenen Faktoren betr�gt dann $n * (n-1) * (n-2) = n! = 3! = 6$. Um dies zu verhindern, werden im Folgenden m�gliche L�sungsans�tze vorgeschlagen:

\begin{enumerate}
	\item Salt beim Hashing verwenden: Durch das Hinzuf�gen einer zuf�llig gew�hlten Zeichenfolge (Salt) an jeden Faktor ist es einem Angreifer nicht mehr m�glich, den ben�tigten Hashwert nur auf Basis des Faktors (d. h. ohne Kenntnisse �ber den Salt) zu berechnen.
	\item Zahlen mit Padding auff�llen: Aufgrund der Verwendung von SHA-256 entspricht der Hashwert und somit auch die daraus abgeleitete Zahl der Gr��enordnung von 256 Bit. Aus sicherheitstechnischen Gr�nden macht es durchaus Sinn, diese Bitl�nge �ber das Hinzuf�gen eines Paddings zu erh�hen. Einem Angreifer ist es dadurch unm�glich, im Nachhinein das Geheimnis zu ermitteln, selbst wenn alle Faktoren bekannt sind. Zudem kann �ber ein Padding die Gesamtl�nge des Geheimnisses gesteuert werden. Je l�nger das Geheimnis ist, desto gr��er muss die gew�hlte Primzahl sein. Auf die Bedingungen und Auswirkungen dieser Primzahl wird im n�chsten Abschnitt eingegangen.
\end{enumerate}

Beim Thema Randomisierung ist zudem wichtig zu erw�hnen, dass in einer realen Anwendung die Koeffizienten $m_i$ f�r $i > 0$ echt zuf�llig und entsprechend der Gleichverteilung aus $\mathbb{F}_p$ gew�hlt werden m�ssen (vgl. Zeile 2 in \autoref{code:create_shares}).

\subsection{Wahl der richtigen Primzahl}
Die Anforderung an die Primzahl $p \geq max(2*r, n + 1)$, wobei $r$ die Bitl�nge des Geheimnisses $S$ repr�sentiert, stellt sicher, dass das Sicherheitsniveau des Verfahrens mindestens die Bitl�nge des zu sch�tzenden Geheimnisses ist und $n$ Shares daraus erzeugt werden k�nnen. Weiter erreicht das Secret-Sharing-Schema von A. Shamir laut dem Bundesamt f�r Sicherheit in der Informationstechnik (BSI) informationstheoretische Sicherheit, was bedeutet, dass ein Angreifer mit unbegrenzten Ressourcen nicht in der Lage ist, das Geheimnis ohne Kenntnis �ber alle $k$ Shares zu rekonstruieren. Die Sicherheit des Verfahrens ist daher nur von der Geheimhaltung der Shares abh�ngig, daher muss "[j]egliche Kommunikation �ber die Teilgeheimnisse [...] verschl�sselt und authentisiert stattfinden, soweit es einem Angreifer physikalisch m�glich ist, diese Kommunikation aufzuzeichnen oder zu manipulieren." \autocite[S. 66]{bsi-richtlinie-2023}.

\subsection{Umgang mit kritischen Daten}
Der Begriff \textquote{Umgang} bezieht sich insbesondere auf die Eingabe, Verarbeitung, �bertragung und Speicherung von sch�tzenswerten Daten. Zu diesen kritischen Daten geh�ren:

\begin{enumerate}
	\item Faktoren (Passwort, Fingerabdruck und Wiederherstellungsschl�ssel): Diese Daten werden f�r die Erstellung des Geheimnisses und zur Authentifizierung ben�tigt. Erh�lt ein Angreifer Zugang zu zwei der drei Informationen, kann dieser sich authentifizieren.
	\item Geheimnis: �hnlich wie bei den Faktoren ist auch das Geheimnis ein kritisches Datum. Durch Verwendung von $k-1$ Anteilen und des Geheimnisses ist es m�glich, das Polynom eindeutig zu rekonstruieren. Zudem k�nnen mithilfe dieser Informationen alle restlichen Anteile generiert werden, sofern die $x$-Koordinaten bekannt sind. Falls jedoch weniger als $k-1$ Anteile vorhanden sind, l�sst sich das Polynom selbst mit dem Geheimnis nicht rekonstruieren.
	\item Shares: Wie im letzten Abschnitt bereits durch den BSI erw�hnt, gilt es die Shares geheim zu halten. Um sicherzugehen, dass eine Nachricht nicht w�hrend der �bertragung von dem erwarteten Sender stammt und diese nicht auf dem Weg manipuliert wurde, k�nnen beispielsweise Message Authentication Codes (MAC) oder Signaturverfahren eingesetzt werden. 
\end{enumerate}

Bei all diesen Daten gilt es zu beachten, dass die damit verbundenen Informationen vor einer �bertragung oder Speicherung unkenntlich gemacht werden. Dies kann zum Beispiel durch die Anwendung einer Hashfunktion oder einer Verschl�sselung passieren.

\subsection{Verwendung einer kryptographisch starken Hashfunktion}
Die in der Implementierung verwendete Hashfunktion SHA-256 ist ein Beispiel einer kryptographisch starken Hashfunktion. Kryptographisch stark bedeutet laut einer Definition des BSI \autocite[S. 46]{bsi-richtlinie-2023}, dass es praktisch nicht m�glich ist ...

\begin{itemize}
	\item ... f�r ein gegebenes $h \in \lbrace 0, 1 \rbrace ^n$ einen Wert $m \in \lbrace 0, 1 \rbrace ^*$ mit $H(m) = h$ zu finden (\textit{Einweg-Eigenschaft}).
	\item ... f�r ein gegebenes $m \in \lbrace 0, 1 \rbrace ^*$ einen Wert $m' \in \lbrace 0, 1 \rbrace ^* \textbackslash \lbrace m \rbrace $ mit $H(m) = H(m')$ zu finden (\textit{2nd-Preimage-Eigenschaft}).
	\item ... zwei Werte $m, m' \in \lbrace 0, 1 \rbrace ^*$ mit $m \neq m'$ und $H(m) = H(m')$ zu finden (\textit{Kollisionsresistenz}).
\end{itemize}

Weitere Hashfunktionen, die diese Eigenschaften nach heutigem Kennt�nis�stand erf�llen, sind SHA-256, SHA-512/256, SHA-384 und SHA-512 sowie die SHA-3-Familie ab SHA3-256 und k�nnen f�r eine Realisierung verwendet werden. Die Verwendung einer anderen Hashfunktion als SHA-256 wirkt sich auf die Laufzeit aus, da eine Verdopplung der Bitl�nge (SHA-256 vs. SHA-512) der einzelnen Hashwerte auch zu einer Verdopplung der Bitl�nge des Geheimnisses f�hrt, wodurch die Bitl�nge der zu berechnenden Primzahl ebenfalls doppelt so gro� sein muss. Die konkreten Unterschiede werden im n�chsten Kapitel genauer betrachtet.
\section{Benchmarks}
\lipsum[1]

% Ausprobieren:
%	- SHA512 statt SHA256
%	- Gro�e Primzahl (3000 Bits)
\section{Anwendung in der digitalen Welt}
Die Anwendung des Domain-Name-Systems (DNS) hat eine zentrale Bedeutung in der digitalen Welt erlangt, da es die Br�cke zwischen menschenlesbaren Domainnamen wie \codeword{oth-aw.de} und maschinenlesbaren IP-Adressen wie \codeword{195.37.42.173} bildet. Ohne ausreichende Sicherheitsvorkehrungen besteht f�r einen Angreifer die M�glichkeit, sich als ein solches DNS-System auszugeben und eine falsche IP-Adresse f�r eine bestimmte Domain zur�ckzuliefern, was erhebliche Folgen haben kann. Um dies zu verhindern, nutzt die daf�r verantwortliche Non-Profit-Organisation ICANN Kryptographie. ICANN greift dabei auf das Prinzip von Secret Sharing zur�ck, indem der Master-Schl�ssel in insgesamt sieben Shares aufgeteilt und auf Smartcards an sieben Personen mit unterschiedlichen geografischen Standorten verteilt wird. Das dabei verwendete (5, 7)-Schwellenwertschema sagt aus, dass f�nf der sieben Personen zusammenkommen m�ssen, um auf das Geheimnis, den Master-Schl�ssel, zugreifen zu k�nnen \autocite{rosulek-icann-2017}. Dieses Anwendungsszenario zeigt das Vertrauen in Secret Sharing und best�tigt, dass sicherheitsrelevante Dienste von diesem Verfahren profitieren k�nnen.

\subsection{Beispielanwendung im Web- und App-Bereich}
Eine Besonderheit der obigen Implementierung ist, dass der Benutzer zu keinem Zeitpunkt direkten Zugriff auf die Shares hat. Jedes Share wird erst nach korrekten Eingabe des dazugeh�rigen Faktors freigegeben.

% Zusammenspiel aus Webanwendung und App	
%	- Szenario durchspielen (Nutzerregistrierung, Speicherung der Shares, Authentifizierung)

% Literaturverzeichnis
\printbibliography

\end{document}