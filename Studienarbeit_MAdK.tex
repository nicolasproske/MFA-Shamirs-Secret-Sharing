% Praeamble
\documentclass[conference,10pt,a4paper]{IEEEtran}

\usepackage[T1]{fontenc}
\usepackage[ansinew]{inputenc}
\usepackage[ngerman]{babel}
\usepackage{lipsum} % Zum Generieren von Blindtext
\usepackage{csquotes} % Anf�hrungszeichen

\usepackage{amsmath}
\usepackage{pgfplots}
\pgfplotsset{compat=1.16}
\usepackage{graphicx}

\usepackage{hyperref}
\usepackage[figure]{hypcap}

\renewcommand\IEEEkeywordsname{Schl�sselw�rter}
\def\code#1{\texttt{#1}}

% Literaturverzeichnis
\usepackage[style=ieee,backend=biber]{biblatex}
\addbibresource{assets/literature.bib}
\renewcommand*{\bibfont}{\raggedright\small}

\begin{document}

% Angaben
\title{Analyse und Implementierung einer Multi-Faktor-Authentifizierung mit \\ Shamir's Secret Sharing}
\author{
	\IEEEauthorblockN{Nicolas Proske}
	\IEEEauthorblockA{
		\textit{Ostbayerische Technische Hochschule Amberg-Weiden} \\
		Moderne Anwendungen der Kryptographie \\
		E-Mail: \code{n.proske@oth-aw.de} \\
		Matr.-Nr.: \code{87672270}
	}
}

\maketitle

% Abstract
\begin{abstract}
\lipsum[1]
\end{abstract}

% Schluesselwoerter
\begin{IEEEkeywords}
MFA, Secret Sharing
\end{IEEEkeywords}

\section{Einleitung}
\lipsum[1] \autocite{sample-2023}
\section{Authentifizierung mit Faktoren}
\subsection{Ein-Faktor-Authentifizierung}
Lange Zeit haben Authentifizierungsmethoden auf einem einzelnen Identifikationsfaktor beruht, in der Regel einer Kombination aus Benutzername und Passwort. Wenn diese beiden Parameter �ber mehrere Dienste hinweg identisch sind, bedeutet dies, dass ein Angreifer, der ein einziges Konto kompromittiert, automatisch Zugriff auf die anderen Konten erh�lt \textemdash{} Dabei spielt die St�rke des Passworts keine Rolle. Dieser One-Factor-Authentication (OFA) Ansatz  hat jahrzehntelang das R�ckgrat der Informationssicherheit gebildet. Dennoch hat sich angesichts der zunehmenden Komplexit�t von Cyberbedrohungen gezeigt, dass die Abh�ngigkeit von einem einzigen Faktor f�r die Authentifizierung eine Schwachstelle darstellt, die anf�llig f�r verschiedene Verletzungen wie Brute-Force-Angriffe, Phishing und Social Engineering ist. Diese Schwachstellen verdeutlichen, dass OFA f�r heutige Anwendungsf�lle in aller Regel keine ausreichende Sicherheit mehr bietet.

\subsection{Multi-Faktor-Authentifizierung}
Multi-Factor-Authentication (MFA) stellt einen signifikanten Fortschritt in der Evolution der digitalen Sicherheitsma�nahmen dar. Im Gegensatz zur Einzelfaktor-Authentifizierung, die �blicherweise auf einer einzigen Form des Nachweises wie einem Passwort basiert, erh�ht MFA die Sicherheit durch zus�tzliche Schutzebenen, indem mehrere unabh�ngige Zugangsdaten f�r die Authentifizierung erforderlich sind. Diese Zugangsdaten k�nnen in drei Hauptkategorien eingeteilt werden:

\begin{enumerate}
	\item \textit{Wissen}: Informationen, die der Benutzer kennt, wie Passw�rter, PINs und Antworten auf geheime Fragen.
  	\item \textit{Eigenheit}: Biologische Merkmale, die einzigartig f�r den Benutzer sind, wie Fingerabdr�cke, Netzhautmuster oder Gesichtserkennung.
  	\item \textit{Besitz}: Gegenst�nde oder Ger�te, die der Benutzer besitzt, wie Smartphones, Chipkarten oder physische Schl�ssel. Die Best�tigung des Besitzes kann verschiedene Formen annehmen, angefangen von der Entgegennahme und Eingabe eines per SMS an eine registrierte Telefonnummer gesendeten Codes bis hin zum Einsetzen eines physischen Schl�ssels in ein Schloss.
\end{enumerate}

Der Hauptvorteil von MFA gegen�ber OFA liegt daher im schichtbasierten Ansatz. Selbst wenn ein Angreifer es schafft, einen Authentifizierungsfaktor zu umgehen, bieten die verbleibenden Faktoren weiterhin Schutz. Eine Kompromittierung eines Faktors gef�hrdet also nicht die Gesamtsicherheit. Trotz der St�rken bringt eine MFA auch eigene Herausforderungen mit sich, wie beispielsweise die potenziell erh�hte Komplexit�t und die Notwendigkeit f�r Benutzer, mehrere Authentifizierungsfaktoren zu verwalten. Dennoch �berwiegen die Vorteile der MFA oft diese potenziellen Nachteile, insbesondere in Umgebungen, in denen der Schutz sensibler Daten oberste Priorit�t hat.
\section{Shamir's Secret Sharing}
Secret Sharing ist ein grundlegender Baustein der modernen Kryptographie. Eines der bekanntesten Verfahren wurde am 1. November 1979 ver�ffentlicht und ist nach seinem Erfinder Adi Shamir, einem israelischen Kryptographen, benannt: Shamir's Secret Sharing \autocite{shamir-secretsharing-1979}.\par

Es basiert auf der Idee, ein Geheimnis in mehrere Teile, sogenannte Shares, aufzuteilen. Um das Geheimnis wiederherzustellen, m�ssen eine bestimmte Anzahl dieser Shares zusammengebracht werden. Jeder einzelne Share ist f�r sich genommen bedeutungslos und gibt keinerlei Informationen preis. Ein Schwellenwert definiert die minimale Anzahl von Shares, die erforderlich sind, um das Geheimnis wiederherstellen zu k�nnen. Dies stellt sicher, dass das Geheimnis selbst dann sicher bleibt, wenn ein Teil der Shares verloren gehen oder in die H�nde eines Angreifers gelangen. Das dabei verwendete (k, n)-Schwellenwertschema legt fest, wie viele $k$ Shares ben�tigt werden, um auf das Geheimnis zu kommen, $n$ ist gr��er $k$ und bezieht sich auf die Gesamtzahl der Shares, in die das Geheimnis aufgeteilt wird.

\subsection{Mathematische Veranschaulichung}
Shamir's Secret Sharing basiert auf dem Prinzip der Polynominterpolation in endlichen K�rpern, wobei $k$ Punkte ein Polynom vom Grad $k - 1$ eindeutig definieren. Um dies anhand eines mathematischen Beispiels zu veranschaulichen, wird im Folgenden ein (2, 3)-Schwellenschema mit $k = 2$ und $n = 3$ betrachtet, bei dem das Geheimnis $S$ der Zahl $42$ entspricht. Sei $p = 43$ eine Primzahl mit $p > S$. Alle Berechnungen erfolgen im endlichen K�rper $\mathbb{F}_p$.

\subsubsection{Generierung der Shares}
Der erste Schritt besteht darin, ein Polynom vom Grad $k - 1 = 2 - 1 = 1$ aufzustellen:

\begin{equation*}
f(x) = mx + b \mod{p}
\end{equation*}

Die Konstante $b$ entspricht dabei dem Geheimnis $S$. Aus Gr�nden der �bersichtlichkeit wird in diesem Beispiel $m = 4$ gew�hlt:

\begin{equation*}
f(x) = 4x + 42 \mod{43}
\end{equation*}

Im n�chsten Schritt erfolgt die Berechnung von $n$ Punkten in der Ebene. Dazu wird f�r $x = 1...n$ eingesetzt:

\begin{equation*}
\begin{aligned}
&\text{F�r } x = 1: y_1 = f(1) = 4*1 + 42 \mod{43} = 3 \\
&\text{F�r } x = 2: y_2 = f(2) = 4*2 + 42 \mod{43} = 7 \\
&\text{F�r } x = 3: y_3 = f(3) = 4*3 + 42 \mod{43} = 11
\end{aligned}
\end{equation*}

Jeder der berechneten Punkte $(1, 3), (2, 7), (3, 11)$ repr�sentiert dabei einen Share. \autoref{tab:shares} zeigt die Verteilung aller drei Shares an insgesamt drei unterschiedliche Nutzer:

\begin{table}[!h]
\centering
\normalsize
\begin{tabularx}{0.45\textwidth}{*3{>{\centering\arraybackslash}X}@{}}
\textit{Verteilung an} user($i$) & \textit{$f(i)$} & share($i, f(i) \mod{p}$) \\
\hline
1 & $f(1) = 46$ & (1, 3) \\
2 & $f(2) = 50$ & (2, 7) \\
3 & $f(3) = 54$ & (3, 11) \\
\end{tabularx}
\medskip
\caption{Verteilung der Shares}
\label{tab:shares}
\end{table}

\subsubsection{Rekonstruktion mit linearem Gleichungssystem}
Sind nun $k$ Punkte gegeben, kann das urspr�ngliche Geheimnis rekonstruiert werden. Im Folgenden werden die Punkte $(1, 3)$ und $(2, 7)$ als Gleichungen in einem linearen Gleichungssystem dargestellt:

\begin{equation*}
\begin{aligned}
&\text{1. } m + b = 3 \\
&\text{2. } 2m + b = 7
\end{aligned}
\end{equation*}

Die Unbekannten werden nun �ber das Substitutionsverfahren gel�st. Durch Umstellen der ersten Gleichung nach $b$ folgt $b = 3 - m$. Dieser Ausdruck wird in die zweite Gleichung eingesetzt, was zu $2m + (3 - m) = 7$ f�hrt. Daraus folgt $m = 4$. Die erhaltene L�sung f�r $m$ wird dann in die umgestellte erste Gleichung eingesetzt, um $b$ zu berechnen: $b = 3 - 4 = -1$. Da die Berechnungen im endlichen K�rper $\mathbb{F}_{43}$ durchgef�hrt werden, wird das Ergebnis $\mod{43}$ genommen, um das Geheimnis im Wertebereich von $0$ bis $p-1$ zu erhalten: $b = -1 \mod{43} = 42$, was dem Geheimnis $S = 42$ entspricht. Bei gr��eren Werten von $k$ w�rde ein Polynom h�heren Grades und ein entsprechend gr��eres lineares Gleichungssystem entstehen.

\subsubsection{Rekonstruktion mit Lagrange-Interpolations-Formel}
Die Lagrange-Interpolation ist das in der Praxis am h�ufigsten eingesetzte Verfahren zur Bestimmung des Polynoms einer bestimmten Ordnung, das durch eine gegebene Menge von Punkten verl�uft. Diese Methode bietet den Vorteil, dass sie direkt eine Formel zur Rekonstruktion des Geheimnisses liefert, ohne dass ein Gleichungssystem explizit gel�st werden muss. Die Koeffizienten $m_0, ..., m_{k-1}$ eines unbekannten Polynoms $f$ vom Grad $k-1$ aus $k$ Punkten $(x_i, y_i)$ k�nnen wie folgt berechnet werden:

\[
f(x) = \sum_{i=1}^{k} \left[ y_i \cdot \prod_{\substack{1 \leq j \leq k \\ i \neq j}} \frac{x - x_j}{x_i - x_j} \right] \mod{p}
\]

Unter Verwendung dieser Formel l�sst sich $m_0 = f(0)$ und damit das Geheimnis $S$ aus $k$ gegebenen Punkten berechnen \autocite[S. 65 f.]{bsi-richtlinie-2023}.
\section{Die Idee der Kombination}
Das Ziel dieser Arbeit besteht darin, die Vorteile beider Verfahren zu kombinieren, um eine robuste und sichere Authentifizierungsmethode zu entwickeln. Im Kern wird ein Geheimnis aus mehreren Authentifizierungsfaktoren generiert und mithilfe von Shamir's Secret Sharing aufgeteilt. Die zur Multi-Faktor-Authentifizierung verwendeten Merkmale entsprechen einem Passwort (Wissen), dem Fingerabdruck des Nutzers (Eigenheit) und einem Wiederherstellungsschl�ssel in Form eines QR-Codes (Besitz). Die Kombination aller drei Faktoren dient als Grundlage zur Berechnung des Geheimnisses, welches anschlie�end mittels Shamir's Secret Sharing in drei Shares aufgeteilt wird, wobei nur zwei St�ck zur sp�teren Authentifizierung ben�tigt werden. Jeder Share wird anschlie�end eindeutig zu einem der obigen Faktoren zugeordnet und auf unterschiedlichen Wegen sicher abgelegt, zum Beispiel der Passwort-Share in einer Datenbank, der Fingerabdruck-Share im sicheren Bereich eines Smartphones und der zugeh�rige Share zum Wiederherstellungsschl�ssel als QR-Code ausgedruckt an einem sicheren Ort.

\subsection{Vorteile dieses Konzepts}
\subsubsection{Erh�hte Sicherheit}
MFA und SSS erg�nzen sich gegenseitig, um eine robuste Sicherheitsarchitektur zu schaffen. W�hrend MFA bereits eine zus�tzliche Sicherheitsebene durch die Verwendung mehrerer Faktoren bietet, stellt Shamir's Secret Sharing sicher, dass die zu sch�tzenden Daten unverschl�sselt dezentral gespeichert werden k�nnen, da ein einzelner Share keine R�ckschl�sse auf das Geheimnis zul�sst. Dadurch wird das Risiko eines vollst�ndigen Datenlecks oder unbefugten Zugriffs drastisch minimiert.

\subsubsection{Flexibilit�t}
Benutzer k�nnen aus drei verschiedenen Optionen (Passwort, Fingerabdruck und Wiederherstellungsschl�ssel) zur Authentifizierung w�hlen. Dar�ber hinaus kann die Aufteilung von Shares an unterschiedliche Ger�te oder Personen erfolgen, um den Bed�rfnissen und Anforderungen eines Nutzers gerecht zu werden.

\subsubsection{Schutz vor Datenverlust}
Wenn beispielsweise ein Benutzer sein Smartphone verliert oder es ir�re�pa�ra�bel besch�digt wurde, ist weiterhin ein Zugriff �ber die beiden anderen Faktoren gew�hrleistet. Dies bietet einen zus�tzlichen Schutz vor Datenverlust.

% Literaturverzeichnis
\printbibliography

\end{document}