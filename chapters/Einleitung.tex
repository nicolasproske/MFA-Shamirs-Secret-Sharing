\section{Einleitung}
In der heutigen digitalen Welt, in der eine �berw�ltigende Menge an Daten generiert, gespeichert und �ber verschiedene Plattformen �bertragen wird, ist der Bedarf an Datenschutz und Datensicherheit relevanter denn je. Die mit der Digitalisierung einhergehenden M�glichkeiten bergen ein erhebliches Risiko f�r Datendiebstahl, unbefugten Zugriff und Cyberangriffe. Obwohl laut einer Umfrage \autocite[S. 23]{gen-insights-2023} die H�lfte der Erwachsenen weltweit glauben, dass die von ihnen ergriffenen Ma�nahmen ausreichen, um sich gegen Identit�tsdiebstahl zu sch�tzen, sind 63 Prozent dar�ber besorgt, dass ihre Identit�t gestohlen wird. Weiter f�hlen sich knapp sieben von zehn Menschen heute anf�lliger f�r Identit�tsdiebstahl als noch vor ein paar Jahren. Ein wesentlicher Grund f�r die Zunahme von Identit�tsdiebstahl liegt neben zu schwachen Passw�rtern haupts�chlich daran, wie Menschen damit umgehen. Bei einer Frage bez�glich der mehrmaligen Verwendung derselben Benutzernamen und Passw�rter haben 82 Prozent zugegeben, zumindest manchmal dieselben Anmeldedaten f�r unterschiedliche Konten zu verwenden. Knapp die H�lfte davon, etwa 45 Prozent, verwenden sogar immer oder in den meisten F�llen dieselben Zugangsdaten \autocite[S. 12]{ibm-security-2021}.