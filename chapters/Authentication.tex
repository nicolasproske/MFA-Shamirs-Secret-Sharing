\section{Authentifizierung mit Faktoren}
\subsection{Ein-Faktor-Authentifizierung}
Lange Zeit haben Authentifizierungsmethoden auf einem einzelnen Identifikationsfaktor beruht, in der Regel einer Kombination aus Benutzername und Passwort. Wenn diese beiden Parameter �ber mehrere Dienste hinweg identisch sind, bedeutet dies, dass ein Angreifer, der ein einziges Konto kompromittiert, automatisch Zugriff auf die anderen Konten erh�lt \textemdash{} Dabei spielt die St�rke des Passworts keine Rolle. Dieser One-Factor-Authentication-Ansatz (OFA) hat jahrzehntelang das R�ckgrat der Informationssicherheit gebildet. Dennoch hat sich angesichts der zunehmenden Komplexit�t von Cyberbedrohungen gezeigt, dass die Abh�ngigkeit von einem einzigen Faktor f�r die Authentifizierung eine Schwachstelle darstellt, die anf�llig f�r verschiedene Verletzungen wie Brute-Force-Angriffe, Phishing und Keylogging ist. Diese Schwachstellen verdeutlichen, dass OFA f�r heutige Anwendungsf�lle in aller Regel keine ausreichende Sicherheit mehr bietet.

\subsection{Mehrfaktor-Authentifizierung}
Multi-Factor-Authentication (MFA) stellt einen signifikanten Fortschritt in der Evolution der digitalen Sicherheitsma�nahmen dar. Im Gegensatz zur Einzelfaktor-Authentifizierung, die �blicherweise auf einer einzigen Form des Nachweises wie einem Passwort basiert, erh�ht MFA die Sicherheit durch zus�tzliche Schutzebenen, indem mehrere unabh�ngige Zugangsdaten f�r die Authentifizierung erforderlich sind. Diese Zugangsdaten k�nnen in drei Hauptkategorien eingeteilt werden:

\begin{enumerate}
	\item \textit{Wissen}: Informationen, die der Benutzer kennt, wie Passw�rter, PINs und Antworten auf geheime Fragen.
  	\item \textit{Besitz}: Gegenst�nde oder Ger�te, die der Benutzer besitzt, wie Smartphones, Chipkarten oder physische Schl�ssel. Die Best�tigung des Besitzes kann verschiedene Formen annehmen, angefangen von der Entgegennahme und Eingabe eines per SMS an eine registrierte Telefonnummer gesendeten Codes bis hin zum Einsetzen eines physischen Schl�ssels in ein Schloss.
  	\item \textit{Eigenheit}: Biologische Merkmale, die einzigartig f�r den Benutzer sind, wie Fingerabdr�cke, Netzhautmuster oder Gesichtserkennung.
\end{enumerate}

Der Hauptvorteil von MFA gegen�ber OFA liegt daher im schichtbasierten Ansatz. Selbst wenn ein Angreifer es schafft, einen Authentifizierungsfaktor zu umgehen, bieten die verbleibenden Faktoren weiterhin Schutz. Eine Kompromittierung eines Faktors gef�hrdet also nicht die Gesamtsicherheit. Trotz der St�rken bringt eine MFA auch eigene Herausforderungen mit sich, wie beispielsweise die potenziell erh�hte Komplexit�t und die Notwendigkeit f�r Benutzer, mehrere Authentifizierungsfaktoren zu verwalten. Dennoch �berwiegen die Vorteile der MFA oft diese potenziellen Nachteile, insbesondere in Umgebungen, in denen der Schutz sensibler Daten oberste Priorit�t hat.