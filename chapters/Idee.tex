\section{Die Idee der Kombination}
Das Ziel dieser Arbeit besteht darin, die Vorteile beider Verfahren zu kombinieren, um eine robuste und sichere Authentifizierungsmethode zu entwickeln. Im Kern wird ein Geheimnis aus mehreren Authentifizierungsfaktoren generiert und mithilfe von Shamir's Secret Sharing aufgeteilt. Die zur Multi-Faktor-Authentifizierung verwendeten Merkmale entsprechen einem Passwort (Wissen), dem Fingerabdruck des Nutzers (Eigenheit) und einem Wiederherstellungsschl�ssel in Form eines QR-Codes (Besitz). Die Kombination aller drei Faktoren dient als Grundlage zur Berechnung des Geheimnisses, welches anschlie�end mittels Shamir's Secret Sharing in drei Shares aufgeteilt wird, wobei nur zwei St�ck zur sp�teren Authentifizierung ben�tigt werden. Jeder Share wird anschlie�end eindeutig zu einem der obigen Faktoren zugeordnet und auf unterschiedlichen Wegen sicher abgelegt, zum Beispiel der Passwort-Share in einer Datenbank, der Fingerabdruck-Share im sicheren Bereich eines Smartphones und der zugeh�rige Share zum Wiederherstellungsschl�ssel als QR-Code ausgedruckt an einem sicheren Ort.

\subsection{Vorteile dieses Konzepts}
\subsubsection{Erh�hte Sicherheit}
MFA und SSS erg�nzen sich gegenseitig, um eine robuste Sicherheitsarchitektur zu schaffen. W�hrend MFA bereits eine zus�tzliche Sicherheitsebene durch die Verwendung mehrerer Faktoren bietet, stellt Shamir's Secret Sharing sicher, dass die zu sch�tzenden Daten unverschl�sselt dezentral gespeichert werden k�nnen, da ein einzelner Share keine R�ckschl�sse auf das Geheimnis zul�sst. Dadurch wird das Risiko eines vollst�ndigen Datenlecks oder unbefugten Zugriffs drastisch minimiert.

\subsubsection{Flexibilit�t}
Benutzer k�nnen aus drei verschiedenen Optionen (Passwort, Fingerabdruck und Wiederherstellungsschl�ssel) zur Authentifizierung w�hlen. Dar�ber hinaus kann die Aufteilung von Shares an unterschiedliche Ger�te oder Personen erfolgen, um den Bed�rfnissen und Anforderungen eines Nutzers gerecht zu werden.

\subsubsection{Schutz vor Datenverlust}
Wenn beispielsweise ein Benutzer sein Smartphone verliert oder es ir�re�pa�ra�bel besch�digt wurde, ist weiterhin ein Zugriff �ber die beiden anderen Faktoren gew�hrleistet. Dies bietet einen zus�tzlichen Schutz vor Datenverlust.