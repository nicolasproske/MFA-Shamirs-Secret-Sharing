\section{Die Idee der Kombination}
Das Ziel dieser Studienarbeit besteht darin, die Vorteile beider Verfahren zu kombinieren, um eine robuste und sichere Authentifizierungsmethode zu entwickeln.

\begin{figure}[!h]
\centering
\begin{tikzpicture}

\node [data] at (0,-1) (input_factors) {Nutzer legt drei Faktoren fest};
\node [process] at (0,-2) (transform_factors) {Faktoren in Geheimnis umwandeln};
\node [process] at (0,-3) (gen_shares) {Geheimnis in Shares aufteilen};
\node [process] at (0,-4) (store_shares) {Shares verteilt speichern};

\path [connector] (input_factors) -- (transform_factors);
\path [connector] (transform_factors) -- (gen_shares);
\path [connector] (gen_shares) -- (store_shares);

\end{tikzpicture}
\caption{�berblick: Generierung der Shares}
\label{fig:ueberblick_gen_shares}
\end{figure}

\autoref{fig:ueberblick_gen_shares} zeigt den groben Ablauf, wie die Shares erzeugt werden. Im Kern wird ein Geheimnis aus mehreren Authentifizierungsfaktoren generiert und mit Hilfe von Shamir's Secret Sharing aufgeteilt. Die zur Multi-Faktor-Authentifizierung verwendeten Faktoren entsprechen einem Passwort (Wissen), dem Fingerabdruck des Nutzers (Eigenheit) und einem Wiederherstellungsschl�ssel in Form eines QR-Codes (Besitz). Die Kombination aller drei Faktoren dient als Grundlage zur Berechnung des Geheimnisses, welches anschlie�end mittels Shamir's Secret Sharing in drei Shares aufgeteilt wird, wobei nur zwei St�ck zur sp�teren Authentifizierung ben�tigt werden. Jeder Share wird anschlie�end eindeutig zu einem der obigen Faktoren zugeordnet und auf unterschiedlichen Wegen sicher abgelegt, zum Beispiel der Passwort-Share in einer Datenbank, der Fingerabdruck-Share im sicheren Bereich eines Smartphones und der zugeh�rige Share zum Wiederherstellungsschl�ssel als QR-Code ausgedruckt an einem sicheren Ort.\par

Nachdem alle Shares erfolgreich generiert wurden und gespeichert sind, kann sich der Nutzer mit zwei der drei zu Beginn festgelegten Faktoren authentifizieren. Mit Blick auf \autoref{fig:ueberblick_auth_shares} w�hlt der Nutzer anfangs aus, welche zwei Faktoren er dazu verwenden m�chte. Nach Eingabe eines korrekten Faktors gibt das System den verkn�pften Share \textquote{frei}, welcher im Anschluss zur Rekonstruktion verwendet wird. Sobald beide Shares zur Verf�gung stehen wird das Geheimnis wiederhergestellt und �berpr�ft, ob der Wert dem urspr�nglichen Geheimnis entspricht. Falls ja, ist die Authentifizierung erfolgreich, andernfalls erh�lt der Nutzer eine Fehlermeldung.

\begin{figure}[!h]
\centering
\begin{tikzpicture}

\node [data] at (0,-1) (def_factors) {Nutzer w�hlt zwei Faktoren aus};
\node [data] at (0,-2) (input_factors) {Nutzer gibt Faktoren ein};
\node [process] at (0,-3) (recon_shares) {Geheimnis rekonstruieren};
\node [decision] at (0,-5) (correct) {Korrekt?};

\node[draw=none] at (1.625, -4.75) (no) {Nein};
\node[draw=none] at (0.35, -6.25) (yes) {Ja};

\node [process] at (3,-5) (error) {Fehler};
\node [process] at (0,-7) (success) {Authentifiziert};

\path [connector] (def_factors) -- (input_factors);
\path [connector] (input_factors) -- (recon_shares);
\path [connector] (recon_shares) -- (correct);
\path [connector] (correct) -- (error);
\path [connector] (correct) -- (success);

\end{tikzpicture}
\caption{�berblick: Authentifizierung mit Shares}
\label{fig:ueberblick_auth_shares}
\end{figure}

\subsection{Vorteile}
\subsubsection{Erh�hte Sicherheit}
MFA und SSS erg�nzen sich gegenseitig, um eine robuste Sicherheitsarchitektur zu schaffen. W�hrend MFA bereits eine zus�tzliche Sicherheitsebene durch die Verwendung mehrerer Faktoren bietet, stellt Shamir's Secret Sharing sicher, dass die zu sch�tzenden Daten unverschl�sselt dezentral gespeichert werden k�nnen, da ein einzelner Share keine R�ckschl�sse auf das Geheimnis zul�sst. Dadurch wird das Risiko eines vollst�ndigen Datenlecks oder unbefugten Zugriffs drastisch minimiert.

\subsubsection{Flexibilit�t}
Benutzer k�nnen aus drei verschiedenen Optionen (Passwort, Fingerabdruck und Wiederherstellungsschl�ssel) zur Authentifizierung w�hlen. Dar�ber hinaus kann die Aufteilung von Shares an unterschiedliche Ger�te oder Personen erfolgen, um sowohl den Bed�rfnissen und Anforderungen des Nutzers als auch eines Systems gerecht zu werden.

\subsubsection{Schutz vor Datenverlust}
Wenn beispielsweise ein Benutzer das verkn�pfte Smartphone verliert oder es ir�re�pa�ra�bel besch�digt wurde, ist weiterhin ein Zugriff �ber die beiden anderen Faktoren gew�hrleistet. Dies bietet einen zus�tzlichen Schutz vor Datenverlust.\par

Zusammenfassend l�sst sich sagen, dass das Zusammenspiel beider Methoden die Sicherheit eines Authentifizierungsprozesses erh�ht. Deshalb wird im Folgenden eine m�gliche Implementierung vorgestellt, welche die einzelnen Schritte im Detail verdeutlichen soll.