\section{Anwendung in der digitalen Welt}
Die Anwendung des Domain-Name-Systems (DNS) hat eine zentrale Bedeutung in der digitalen Welt erlangt, da es die Br�cke zwischen menschenlesbaren Domainnamen wie \codeword{oth-aw.de} und maschinenlesbaren IP-Adressen wie \codeword{195.37.42.173} bildet. Ohne ausreichende Sicherheitsvorkehrungen besteht f�r einen Angreifer die M�glichkeit, sich als ein solches DNS-System auszugeben und eine falsche IP-Adresse f�r eine bestimmte Domain zur�ckzuliefern, was erhebliche Folgen haben kann. Um dies zu verhindern, nutzt die daf�r verantwortliche Non-Profit-Organisation ICANN Kryptographie. ICANN greift dabei auf das Prinzip von Secret Sharing zur�ck, indem der Master-Schl�ssel in insgesamt sieben Shares aufgeteilt und auf Smartcards an sieben Personen mit unterschiedlichen geografischen Standorten verteilt wird. Das dabei verwendete (5, 7)-Schwellenwertschema sagt aus, dass f�nf der sieben Personen zusammenkommen m�ssen, um auf das Geheimnis, den Master-Schl�ssel, zugreifen zu k�nnen \autocite{rosulek-icann-2017}. Dieses Anwendungsszenario zeigt das Vertrauen in Secret Sharing und best�tigt, dass sicherheitsrelevante Dienste von diesem Verfahren profitieren k�nnen.

\subsection{Beispielanwendung im Web- und App-Bereich}
Eine Besonderheit der obigen Implementierung ist, dass der Benutzer zu keinem Zeitpunkt direkten Zugriff auf die Shares hat. Jedes Share wird erst nach korrekten Eingabe des dazugeh�rigen Faktors freigegeben.

% Zusammenspiel aus Webanwendung und App	
%	- Szenario durchspielen (Nutzerregistrierung, Speicherung der Shares, Authentifizierung)