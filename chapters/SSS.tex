\section{Shamir's Secret Sharing}
Secret Sharing ist ein grundlegender Baustein der modernen Kryptographie. Eines der bekanntesten Verfahren wurde am 1. November 1979 ver�ffentlicht und ist nach seinem Erfinder Adi Shamir, einem israelischen Kryptographen, benannt: Shamir's Secret Sharing \autocite{shamir-secretsharing-1979}.\par

Es basiert auf der Idee, ein Geheimnis in mehrere Teile, sogenannte Shares, aufzuteilen. Um das Geheimnis wiederherzustellen, m�ssen eine bestimmte Anzahl dieser Shares zusammengebracht werden. Jeder einzelne Share ist f�r sich genommen bedeutungslos und gibt keinerlei Informationen preis. Ein Schwellenwert definiert die minimale Anzahl von Shares, die erforderlich sind, um das Geheimnis wiederherstellen zu k�nnen. Dies stellt sicher, dass das Geheimnis selbst dann sicher bleibt, wenn ein Teil der Shares verloren gehen oder in die H�nde eines Angreifers gelangen. Das dabei verwendete (k, n)-Schwellenwertschema legt fest, wie viele $k$ Shares ben�tigt werden, um auf das Geheimnis zu kommen, $n$ ist gr��er $k$ und bezieht sich auf die Gesamtzahl der Shares, in die das Geheimnis aufgeteilt wird.

\subsection{Mathematische Veranschaulichung}
Shamir's Secret Sharing basiert auf dem Prinzip der Polynominterpolation in endlichen K�rpern, wobei $k$ Punkte ein Polynom vom Grad $k - 1$ eindeutig definieren. Um dies anhand eines mathematischen Beispiels zu veranschaulichen, wird im Folgenden ein (2, 3)-Schwellenschema mit $k = 2$ und $n = 3$ betrachtet, bei dem das Geheimnis $S$ der Zahl $42$ entspricht. Sei $p = 43$ eine Primzahl mit $p > S$. Alle Berechnungen erfolgen im endlichen K�rper $\mathbb{F}_p$.

\subsubsection{Generierung der Shares}
Der erste Schritt besteht darin, ein Polynom vom Grad $k - 1 = 2 - 1 = 1$ aufzustellen:

\begin{equation*}
f(x) = mx + b \mod{p}
\end{equation*}

Die Konstante $b$ entspricht dabei dem Geheimnis $S$. Aus Gr�nden der �bersichtlichkeit wird in diesem Beispiel $m = 4$ gew�hlt:

\begin{equation*}
f(x) = 4x + 42 \mod{43}
\end{equation*}

Im n�chsten Schritt erfolgt die Berechnung von $n$ Punkten in der Ebene. Dazu wird f�r $x = 1...n$ eingesetzt:

\begin{equation*}
\begin{aligned}
&\text{F�r } x = 1: y_1 = f(1) = 4*1 + 42 \mod{43} = 3 \\
&\text{F�r } x = 2: y_2 = f(2) = 4*2 + 42 \mod{43} = 7 \\
&\text{F�r } x = 3: y_3 = f(3) = 4*3 + 42 \mod{43} = 11
\end{aligned}
\end{equation*}

Jeder der berechneten Punkte $(1, 3), (2, 7), (3, 11)$ repr�sentiert dabei einen Share. \autoref{tab:shares} zeigt die Verteilung aller drei Shares an insgesamt drei unterschiedliche Nutzer:

\begin{table}[!h]
\centering
\normalsize
\begin{tabularx}{0.45\textwidth}{*3{>{\centering\arraybackslash}X}@{}}
\textit{Verteilung an} user($x$) & \textit{$f(x)$} & share($x, f(x) \mod{p}$) \\
\hline
1 & $f(1) = 46$ & (1, 3) \\
2 & $f(2) = 50$ & (2, 7) \\
3 & $f(3) = 54$ & (3, 11) \\
\end{tabularx}
\medskip
\caption{Verteilung der Shares}
\label{tab:shares}
\end{table}

\subsubsection{Rekonstruktion mit linearem Gleichungssystem}
Sind nun $k$ Punkte gegeben, kann das urspr�ngliche Geheimnis rekonstruiert werden. Im Folgenden werden die Punkte $(1, 3)$ und $(2, 7)$ als Gleichungen in einem linearen Gleichungssystem dargestellt:

\begin{equation*}
\begin{aligned}
&\text{1. } m + b = 3 \\
&\text{2. } 2m + b = 7
\end{aligned}
\end{equation*}

Die Unbekannten werden nun �ber das Substitutionsverfahren gel�st. Durch Umstellen der ersten Gleichung nach $b$ folgt $b = 3 - m$. Dieser Ausdruck wird in die zweite Gleichung eingesetzt, was zu $2m + (3 - m) = 7$ f�hrt. Daraus folgt $m = 4$. Die erhaltene L�sung f�r $m$ wird dann in die umgestellte erste Gleichung eingesetzt, um $b$ zu berechnen: $b = 3 - 4 = -1$. Da die Berechnungen im endlichen K�rper $\mathbb{F}_{43}$ durchgef�hrt werden, wird das Ergebnis $\mod{43}$ genommen, um das Geheimnis im Wertebereich von $0$ bis $p-1$ zu erhalten: $b = -1 \mod{43} = 42$, was dem Geheimnis $S = 42$ entspricht. Bei gr��eren Werten von $k$ w�rde ein Polynom h�heren Grades und ein entsprechend gr��eres lineares Gleichungssystem entstehen.

\subsubsection{Rekonstruktion mit Lagrange-Interpolations-Formel}
Die Lagrange-Interpolation ist das in der Praxis am h�ufigsten eingesetzte Verfahren zur Bestimmung des Polynoms einer bestimmten Ordnung, das durch eine gegebene Menge von Punkten verl�uft. Diese Methode bietet den Vorteil, dass sie direkt eine Formel zur Rekonstruktion des Geheimnisses liefert, ohne dass ein Gleichungssystem explizit gel�st werden muss. Die Koeffizienten $m_0, ..., m_{k-1}$ eines unbekannten Polynoms $f$ vom Grad $k-1$ aus $k$ Punkten $(x_i, y_i)$ k�nnen wie folgt berechnet werden:

\[
f(x) = \sum_{i=1}^{k} \left[ y_i \cdot \prod_{\substack{1 \leq j \leq k \\ i \neq j}} \frac{x - x_j}{x_i - x_j} \right] \mod{p}
\]

Unter Verwendung dieser Formel l�sst sich $m_0 = f(0)$ und damit das Geheimnis $S$ aus $k$ gegebenen Punkten berechnen \autocite[S. 65 f.]{bsi-richtlinie-2023}.